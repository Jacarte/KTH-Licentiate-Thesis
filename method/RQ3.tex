
\section{RQ3. To what extent can the artificially generated variants be used to enforce security on Edge-Cloud platforms?}

\newcommand{\mewe}{MEWE\xspace}



In the last research question, we study whether the created variants can be used in real-world applications and what properties offer the composition of the variants as multivariant execution binaries. For this purpose, we build multivariant binaries to be deployed at the Edge in the Fastly platform. We use the variants generated for the programs of the \corpussodium and \corpusqrcode corpora, $2 + 5$ programs involving $ \libsodiumfunctions + \qrcodefunctions$ functions respectively. For this research question, 7 Multivariant binaries are created by converting each program's population into a single function for which each call at runtime selects and executes a different variant. One of the contributions of this work is \mewe, a tool that automatically creates multivariant binaries out of program variants generated by CROW. We simplify the protocol to answer RQ3 in \autoref{diagrams:protocol:rq3}.



\begin{figure*}[h]
    \centering
    \includegraphics[height=2.50in]{diagrams/Rq3.pdf}
    \caption{Multivariant binary creation and workflow for RQ3.}
    \label{diagrams:protocol:rq3}
\end{figure*}

We assess the ability of \mewe to produce binaries that exhibit random execution paths when executed on the edge. We check the diversity of execution traces gathered from the execution of a multivariant binary. The traces are collected from all edge nodes to assess Multivariant Execution (MVE) worldwide. \mewe generates binaries that embed a multivariant behavior. Finally, we measure how \mewe generates different execution times on the edge. Then, we discuss how multivariant binaries contribute to less predictable timing side-channels.

\subsection{Metrics}

To compare the diversity of function traces for the 7 created multivariant binaries, we enunciate the following metrics.  


\begin{metric}{Unique traces: $R(n, e)$.}\label{metric:ratio:mve}
    Let $S(n, e)=\{T_1, T_2, ..., T_{100}\}$ be the collection of 100 traces collected for one endpoint $e$ on an edge node $n$, $H(n, e)$ the collection of hashes of each trace and $U(n, e)$ the set of unique trace hashes in $H(n,e)$.
    The uniqueness ratio of traces collected for edge node $n$ and endpoint $e$ is defined as
    $$
        R(n,e) = \frac{|U(n,e)|}{|H(n, e)|}
    $$
\end{metric}


\begin{metric}{Normalized Shannon entropy: $E(e)$}\label{metric:entropy}
    Let $e$ be an endpoint, $C(e)=\cdot_{n=0}^{64} H(n, e)$ be the union  of all trace hashes for all edge nodes.
    The normalized Shannon Entropy for the endpoint $e$ over the collected traces is defined as: \\
    $$
        E(e)=-\Sigma \frac{p_x*log(p_x)}{log(|C(e)|)}
    $$
    Where $p_x$ is the discrete probability of the occurrence of the hash $x$ over $C(e)$.
    
\end{metric}

Notice that we normalize the standard definition of the Shannon Entropy, \autoref{metric:entropy}, by using the perfect case where all trace hashes are different. 
This normalization allows us to compare the calculated entropy between endpoints.
The value of the metric can go from 0 to 1. The worst entropy, value 0, means that the endpoint always perform the same path independently of the edge node and the number of times the trace is collected for the same node. On the contrary, 1 for the best entropy, when each edge node executes a different path every time the endpoint is requested.

\subsection{Protocol}


We run the experiments to answer RQ3 on the Fastly edge computing platform. We deploy and execute the original and the multivariant endpoints on 64 edge nodes located around the world\footnote{The number of nodes provided in the whole platform is 72, we decided to keep only the 64 nodes that remained stable during our experimentation.}.
These edge nodes usually have an arbitrary and heterogeneous composition in architecture and CPU model.


We execute each endpoint multiple times on each node to measure the diversity of execution traces exhibited by the multivariant binaries. Each query on the same endpoint is performed with the same input value. This guarantees that if we observe different traces for different executions, it is due to the presence of multiple function variants. 
The inputs that we pass to execute the endpoints at the edge and the received output for all executions are available in the reproduction repository at \todo{REPO}. 
% The input values are available as part of our reproduction package.

For each query, we collect the execution trace, i.e.,  the sequence of function names that have been executed when triggering the query.
We instrument the multivariant binaries to record each function entrance to observe these traces.

We then measure the number of unique execution traces exhibited by each multivariant binary, \autoref{metric:ratio:mve}, on each separate edge node. Then, to compare the traces, we hash them with the \texttt{md5sum} function.
We then calculate the number of unique hashes among the 100 traces collected for an endpoint on one edge node.
We follow by collecting the normalized Shannon entropy, \autoref{metric:entropy}, for all collected execution traces for each endpoint.
The Shannon Entropy gives the uncertainty in the outcome of a sampling process.
If a specific trace has a high frequency of appearing in part of the sampling, then it is inevitable that this trace will appear in the other part of the sampling.


We calculate \autoref{metric:entropy} for the 7 endpoints, for 100 traces collected from 64 edge nodes, for a total of 6400 collected traces per endpoint.
Each trace is collected in a round-robin strategy, i.e., the traces are collected from the 64 edge nodes sequentially.
For example, we collect the first trace from all nodes before continuing to collect the second trace.
This process is followed until 100 traces are collected from all edge nodes.


In addition, we collect 100k execution times for each binary, both the original and multivariant binaries.
We perform a Mann-Withney U test \cite{mann1947} to compare both execution time distributions. 
If the P-value is lower than 0.05, two compared distributions are different.


\section{Conclusions}

This chapter presents the methodology we follow to answer our three research questions. We first describe and propose the corpora of programs used in this work. We propose to measure the ability of CROW to generate variants out of \py{303  + \libsodiumfunctions + \qrcodefunctions} functions of our corpora. Then, we suggest using the generated variants to study to what extent they offer different observable behavior through static analysis, dynamic analysis, and a variant's preservation study. Finally, we propose a protocol to study the impact of the composition variants in a multivariant binary deployed at the Edge. Nevertheless, we enumerate and enunciate the properties and metrics that might lead us to answer the impact of automatic diversification for \wasm programs. In the next chapter, we present and discuss the results obtained with this methodology.
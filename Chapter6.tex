\chapter{Conclusion and Future Work}
\label{chapter:conclude}

\wasm has become a new tehcnology for web browsers and standalone engines such as the ones used in Edge-Cloud platforms. \wasm is designed with security and sandboxing premises, yet, is still vulnerable.
Besides, since it is a relatively new technology, new vulnerabilities appear in the wild every, a t a higher pace than the adoption of patches and defenses.
Software diversification, as a widely studied field, could be a solution for known and yet-unknown vulnerabilities. However, none research on this field has been conducted for \wasm.

In this work, we propose an automatic approach to generate software diversification for \wasm. 
Our approaches are based on automated software transformations highlighted by an exhaustive literature research.
In addition, we provide the complementary implementation for our approaches, including a generic LLVM superdiversifier that could be used to extend our ideas to other programming languages.
We empirically demonstrate the impact of our approach by providing Randomization and Multivariant Execution for \wasm.
This chapter is organized into two sections. 
In \autoref{conclusions:summary}, we summarizes the main results we found by answering our research questions enunciated in \autoref{chapter:intro}.
Finally, \autoref{future_work} describes potential future work that could extend this dissertation.

\section{Summary of the results}
\label{conclusions:summary}

\section{Future work}
\label{future_work}

One of our previous contributions trigger a CVE\footnote{\url{https://www.fastly.com/blog/defense-in-depth-stopping-a-wasm-compiler-bug-before-it-became-a-problem}} on the code generation component of wasmtime, highlighting that even when the language specification is meant to be secure, the underlying host implementation might not be. 

\todo{Side channel reproduction and study of the impact of side-channel}

\subsection{wasm-mutate future work}
\todo{Obfuscation and data augmentation}



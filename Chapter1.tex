\chapter{Introduction}

\newcommand{\rqone}{RQ1. To what extent can we artifically generate program variants for \wasm?}

\newcommand{\rqtwo}{RQ2. To what extent are the generated variants dynamically different?}
\newcommand{\rqthree}{RQ3. To what extent do the artificial variants exhibit different execution times on Edge-Cloud platforms?}

Write a short introduction here...


\todo{Moved from Chapter 2}
The low presence of defenses implementations for \wasm motivates our work on Software Diversification as a preemptive technique that can help against known and yet unknown vulnerabilities.

\section{Motivation}

\subsection{Why variants ?}

\subsection{Research questions}
\label{intro:definition:rq}


\begin{enumerate}
    \item \rqone
    \item \rqtwo
    \item \rqthree
\end{enumerate}

%T%he main motivation for this research question is that \wasm was adopted in 2017, and it lacks of natural diversity \citationneeded. Moreover, compared to the work of Harrand \etal \citationneeded, in WebAssembly, we cannot use preexisting and different programs to provide diversification. In fact, according to the work of Hilbig \etal \citationneeded, the artificial variants created with one of our works contributes to the half of executable and available \wasm binaries in the wild. 

\section{Contributions}
\chapter{Technical contributions}
\label{chapter:technical}
% / Automatic Diversity for Wasm/ 


% The need for technical contributions...
We aim to create Artificial Software Diversity for \wasm, by providing tools to make the process easier and feasible for developers and researchers. As far as we know, there is no software that provides Artificial Software Diversification for WebAssembly. Therefore, we need to enunciate the engineering foundation to implement the strategies in \autoref{sota:sota}. Our implementations are part of the contributions of this thesis. Concretely, we provide two software artifacts that complement this work. Our approach generate \wasm program variants statically at compile time to provide randomization. Besides, it provides the tooling to generate MVE binaries for WebAssembly.


In this chapter we describe our technical contributions. In \autoref{tech:generic} we enunciate how the current state-of-the-art lead us to contribute with Software Diversification through LLVM. We follow by describing our two contributions and their main technical insights in \autoref{section:crow} and \autoref{section:mewe}. Besides, we describe a new transformation strategy as part of our contributions. 


% What does it solve ?
% The aim of massive-scale software diversity is to make it 

% . However, 

% In this chapter, we describe our threat model, return- and jump-oriented program- ming (see Section 4.2), new classes of code-reuse attacks. Code-reuse attacks are an attack class that is growing in popularity in response to other host-based defenses.
% Then we describe several compiler-based transformations (Section 4.3) and their im- plementation (Section 4.4) in the LLVM 2.9 compiler toolchain. We also provide some metrics (Section 4.5) for evaluating the effectiveness of these techniques for preventing wide-scale deployment of code-reuse attacks

%\todo{Make a real introduction, the paragraphs below is too fast}






\section{Artificial Software Diversity for \wasm}
\label{tech:generic}

The work of Hilbig et al. \cite{Hilbig2021AnES} at 2021 influences our engineering decisions. According to their work LLVM-based compilers created the 70\% of the \wasm binaries in the wild. Therefore, we decided to provide Artificial Sotfware Diversity for \wasm through LLVM. 
Other solutions would have been to diversify at the source code level, or at the \wasm binary level. However, the former would limit the applicability of our work. The latter, will be addressed in future works (\autoref{future_work}).

LLVM is composed by three main components \citationneeded. First, the frontend (compilers such as clang and rustc) converts the program source code to LLVM intermediate representation (LLVM IR). Second, optimization and transformation passes improve the LLVM IR. Third and final, the backend component is in charge of generating the target machine code. Notice that, the LLVM architecture is highly scalable. The machine code translation of LLVM IR might have any number of custom intermediate passes. In \autoref{diagrams:generic} we show the generic workflow followed in our contributions. 
In the context of our work the LLVM architecture is instantiated over all LLVM frontends \step{1}, it adds a Diversifier as a LLVM IR pass \step{2} and uses a custom Wasm  backend \step{3}.
The dashed squares in \autoref{diagrams:generic} wrap the components for which we contribute.

\begin{figure*}[h]
    \includegraphics[width=\linewidth]{diagrams/architecture.pdf}
    \caption{Generic workflow to create \wasm program variants.}
    \label{diagrams:generic}
\end{figure*}


The generic workflow in \autoref{diagrams:generic} starts by receiving the source code from the program to be compiled to \wasm. Then, an LLVM frontend transforms this source code into LLVM IR representation. The resulting LLVM IR is the input for the Diversifier \step{1}.  
% Our diversifier and first contribution
The diversifier generates LLVM IR variants from of the output of the frontend \step{2}. These variants are inputs for our customized Wasm  backend. In our case the LLVM backend is always for \wasm and, it finalizes the creation of the variants \step{3}. 

Our first technical contribution, CROW  \cite{CROW}, includes the implementation of the diversifier for LLVM and the customized \wasm backend. CROW is able to create several \wasm program variants out of a source code program. In \autoref{section:crow} we dissect CROW into more details.
In addition, an orthogonal contribution comes from the generation of LLVM IR variants at Step~\step{2}. Our second contribution, MEWE  \cite{MEWE}, merges and creates multivariant binaries to provide MVE for \wasm \step{4}. In \autoref{section:mewe} we describe MEWE in details.

%Besides, as we discussed previously, our intention is also to study the impact of our contributions in edge computing and distributed systems and the top edge computing execution platforms, e.g. Cloudflare and Fastly, mostly take \wasm binaries as input. 
%LLVM, on the contrary, supports different languages, with a rich ecosystem of frontends it it can reliably be retargeted to \wasm, thanks to the corresponding mature component in the LLVM toolchain. In addition, the LLVM ecosystem as a whole is very active, providing us with many different tools to facilitate our research endeavour.
%In this chapter we summarize the technical details for our two contributions. \autoref{section:crow} we dissect the main components CROW \cite{CROW} implementation. Finally, in \autoref{section:mewe}, we describe the technical details of our second contribution, MEWE \cite{MEWE}.


\section{CROW}
\label{section:crow}

This section describes CROW, our first contribution. CROW is a tool tailored to create semantically equivalent \wasm variants out of a single program, either C/C++ and Rust code or LLVM bitcode.
In \autoref{diagrams:crow}, we describe the workflow of CROW to create program variants.


\subsection*{Overview}

\begin{figure*}[h]
    \includegraphics[width=\linewidth]{diagrams/generation/crow.drawio.pdf}
    \caption{CROW workflow to generate program variants. CROW takes C/C++ source codes or LLVM bitcodes to look for code blocks that can be replaced by semantically equivalent code and generates program variants by combining them.}
    \label{diagrams:crow}
\end{figure*}

Figure \ref{diagrams:crow} highlights the main two stages of the CROW's workflow, \textit{exploration} and \textit{combining}. The workflow starts by compiling the input program into the LLVM bitcode using clang from the source code. During the \emph{exploration} stage, CROW takes an LLVM bitcode and, for its code blocks, produces a collection of code replacements that are functionally equivalent to the original program. In the following, we enunciate the definitions we use along with this work for a code block, functional equivalence, and code replacement. 


\begin{definition}{Block (based on Aho \etal \cite{ahodragonbook}):}\label{def:code-block}
    Let $P$ be a program. A block $B$ is a grouping of declarations and statements in $P$ inside a function $F$. 
\end{definition}


\begin{definition}{Functional equivalence modulo program state (based on Cohen \etal \cite{cohen1993operating}):}
    \label{def:functional-equivalence}
    Let $B_1$ and $B_2$ be two code blocks according to \autoref{def:code-block}. We consider the program state before the execution of the block, $S_i$, as the input and the program state after the execution of the block, $S_o$, as the output. $B_1$ and $B_2$ are functionally equivalent if given the same input $S_i$ both codes produce the same output $S_o$.
\end{definition}

\begin{definition}{Code replacement:}
    \label{def:code-replacement}
    Let $P$ be a program and $T$ a pair of code blocks $(B_1, B_2)$. $T$ is a candidate code replacement if $B_1$ and $B_2$ are both functionally equivalent as defined in \autoref{def:functional-equivalence}.
    Applying $T$ to $P$ means replacing $B_1$ by $B_2$. The application of $T$ to $P$ produces a program variant $P'$ which consequently is functionally equivalent to $P$.     
\end{definition}

We have found the work of Jacob \etal \cite{jacob2008superdiversifier} on superdiversification the best approach to generate artificial diversity at a fine-grained level. 
We use their seminal work to implement CROW based on two main reasons. First, the code replacements generated by this technique outperform diversification strategies based on hand-written transformation rules and it is fully automatic. Second, the existence of a battle tested superoptimizer for LLVM, Souper \cite{Sasnauskas2017Souper:Superoptimizer}. 
We implement the \emph{exploration} stage by retargeting Souper. The main objective of Souper is to find the best (smallest) possible program, we modify it to keep the intermediate solutions in their searching algorithm to generate program variants.  
We prevent the superoptimizer from synthesizing instructions that have no correspondence in the \wasm backend for the sake of reducing the searching space for variants. Besides, we disable the majority of the pruning strategies of Souper for the sake of more variants.
In addition, we also modify the LLVM compiler, by disabling all optimizations in the \wasm backend that could reverse the superoptimizer transformations, such as constant folding and instructions normalization.


CROW operates at the code block level, taking them from the functions defined inside the input LLVM bitcode module. In addition, the retargeted superoptimizer is in charge of finding the potential places in the original code blocks where a replacement can be applied. Finally, we use the enumerative synthesis strategy of the retargeted superoptimizer to generate code replacements.
The code replacements generated through synthesis are verified, according to \autoref{def:functional-equivalence}, by internally using a theorem prover. 

%\todo{We disable cost restrictions and the LLVM backend optimizations...maybe for the assesment RQ ?}

In the \emph{combining} stage, CROW combines the candidate code replacements to generate different LLVM bitcode variants, selecting and merging the code replacements. 
Then, a variant bitcode is compiled into a \wasm binary if requested. Finally, CROW generates the variants from all possible combinations of code replacements as the power set of all code replacements.  

\subsection*{Constant inferring as a new diversification transformation}

\todo{Constant inferring and why is important and novel in our work.}


\subsection*{Example}
\label{section:crow:example}
 Let us illustrate how CROW works with the simple example code in \autoref{CExample}. The \texttt{f} function calculates the value of $2 * x + x$ where \texttt{x} is the input for the function.  CROW compiles this source code and generates the intermediate LLVM bitcode in the left most part of \autoref{example:crow:original:llvm}. CROW potentially finds two code blocks to look for variants, as the right-most part of \autoref{example:crow:original:llvm} shows.

% snippet of code showing the detection of code blocks

 \begin{code}
    \lstset{language=C,caption={C function that calculates the quantity $2x + x$},label=CExample}
    \begin{lstlisting}[style=CStyle]
    int f(int x) { 
        return 2 * x + x; 
    }
    
    int main(void) { return f(10); }
    \end{lstlisting}
    
    \end{code}

 \lstdefinelanguage{LLVM}
    {morekeywords={i32,mul,align,nsw,add,load,store,define,br, ret, shl, ret},
    sensitive=false,
    morecomment=[l]{;},
    morecomment=[s]{;}{;},
    morestring=[b]",
}
\lstdefinestyle{nccode}{
    numbers=left,
    tabsize=4,
    showspaces=false,
    breaklines=true, 
    showstringspaces=false,
    moredelim=**[is][{\btHL[fill=black!10]}]{`}{`},
    moredelim=**[is][{\btHL[fill=celadon!40]}]{!}{!}
}
\lstset{
    language=LLVM,
    style=nccode,
    %basicstyle=\small\ttfamily,
    columns=fullflexible,
    breaklines=true
}


\begin{code}
    \centering
    \captionof{lstlisting}{LLVM's intermediate representation program and its code blocks.}\label{example:crow:original:llvm}
    \lstset{numbers=none}
    \noindent\begin{minipage}[b]{.55\linewidth}
    \centering
    \begin{lstlisting}[xleftmargin=1em,escapechar=?]
    define i32 @f(i32) {

     ?\tikzmarkWS{2}{code block 2}{12.5}{10}{4.5cm}?
     ?\tikzmarkWS{1}{code block 1}{13.5}{4}{3.5cm}?
     %2 = mul nsw i32 %0,2
     %3 = add nsw i32 %0,%2 

     ret i32 %3
    }
    
    define i32 @main() {
     %1 = tail call i32 @f(i32 10)
     ret i32 %1
    }
    \end{lstlisting}
    \end{minipage}\hfill%
    \noindent\begin{minipage}[b]{.4\linewidth}
    \vspace{-8cm}
        \lstdefinestyle{nccode}{
          tabsize=4, 
          showspaces=false,
          breaklines=true, 
          showstringspaces=false,
        moredelim=**[is][{\btHL[fill=black!10]}]{`}{`},
        moredelim=**[is][{\btHL[fill=celadon!40]}]{!}{!}
        }
        \lstset{
            language=LLVM,
            style=nccode,
            columns=fullflexible,
            breaklines=true,
            belowcaptionskip=1pt,
            abovecaptionskip=1pt,
        }
        \vfill%
        \begin{lstlisting}[label={ref:block1} ,name={A},escapechar=?]
    ?\tikzmarkPROBE{5}{bb4}{-4}{10}? 
    %2 = mul nsw i32 %0,2
        \end{lstlisting}
        \begin{lstlisting}[name={B},escapechar=?]
    ?\tikzmarkPROBE{6}{bb4}{-4}{8}? 
    %2 = mul nsw i32 %0,2
    %3 = add nsw i32 %0,%2
        \end{lstlisting}
    \end{minipage}
\end{code}


\begin{tikzpicture}[remember picture,overlay]
%\path (2.north) edge[<-, bend left] (1.north);
%\path[draw, ->] (3.west) edge[<-, bend left] (2.west);
%\path (4.west) edge[<-, bend left] (3.west);
%\path (1.south) edge[<-, bend left] (4.south);

%\path (2.east) edge[<-, bend left, blue] (5.north);
%\path (3.east) edge[<-, bend right, olive] (2.east);
\path (1.east) edge[<-, bend right, black] (5.east);
\path (2.east) edge[<-, bend right, black] (6.east);
%\path (6.east) edge[<-, bend right, black] (3.east);
%\path (9.east) edge[<-, bend right, black] (4.east);
%\path (7.east) edge[<-, bend right, black] (8.east);
%\path (5.south) edge[<-, bend right, blue] (4.east);
%\path (9.north) edge[<-] (8.south);
%\path (5.south) edge[<-, bend left] (9.south);


%\path (10.north) edge[<-, bend left] (11.north);
%\path (11.south) edge[<-, bend left] (10.south);
%\path (7) edge[<-, bend right] (6.east);
%\path (8) edge[<-, bend right] (7.east);
\end{tikzpicture}

    

CROW, in the exploration stage detects 2 code blocks, \texttt{code\_block\_1} and \texttt{code\_block\_2} as the enclosing boxes in the left most part of \autoref{example:crow:original:llvm} show. CROW synthesizes $2 + 1$ candidate code replacements for each code block respectively as the green highlighted lines show in the right most parts of \autoref{example:crow:original:llvm}.
The baseline strategy of CROW is to generate variants out of all possible combinations of the candidate code replacements, \ie uses the power set of all candidate code replacements.

In the example, the power set is the cartesian product of the found candidate code replacements for each code block, including the original ones, as \autoref{example:crow:original:combination} shows. The power set size results in $6$ potential function variants. Yet, the generation stage would eventually generate $4$ variants from the original program. CROW generated 4 statically different Wasm files, as \autoref{example:crow:variants:wasm} illustrates. This gap between the potential and the actual number of variants is a consequence of the redundancy among the bitcode variants when composed into one. In other words, if the replaced code removes other code blocks, all possible combinations having it will be in the end the same program. In the example case, replacing \texttt{code\_block\_2} by \texttt{mul nsw \%0, 3}, turns \texttt{code\_block\_1} into dead code, thus, later replacements generate the same program variants. The rightmost part of \autoref{example:crow:original:combination} illustrates how for three different combinations, CROW produces the same variant. We call this phenomenon a candidate code replacement overlapping.

One might think that a reasonable heuristic could be implemented to avoid such overlapping cases. Instead, we have found it easier and faster to generate the variants with the combination of the replacement and check their uniqueness after the program variant is compiled. This prevents us from having an expensive checking for overlapping inside the CROW code. Still, this phenomenon calls for later optimizations in future works.

\input{snippets/wasm_codes.tex}


\section{MEWE: Multi-variant Execution for WEbAssembly}
\label{section:mewe}

\newcommand{\tool}{MEWE\xspace}
\newcommand{\repourl}{TODO}
% Overview
This section describes MEWE \cite{MEWE}, our second contribution. 
%\termidx{MEWE }is implemented in 942 lines of C++ code.
The core idea of \tool is to synthesize diversified function variants, using CROW, providing execution-path randomization in an MVE. Thus, the goal of \tool is to synthesize \termidx{multivariant } \wasm binaries. %, according to the threat model presented in \autoref{sec:threat-model}. 

The tool generates application-level \termidx{multivariant }binaries, without any change to the operating system or \wasm runtime.
%It uses the LLVM 12.0.0 libraries to extend the LLVM standard linker tool capability with the \termidx{multivariant }generation.
%Per Crane et al. the execution-path randomization is made to hinder side-channel attacks \cite{crane2015thwarting}. 
% All programs are diversified with behavior preservation guarantees according to the design of \termidx{CROW }(\autoref{section:crow}).
\termidx{MEWE }creates an MVE by intermixing functions for which \termidx{CROW }generates variants, as \step{2} in \autoref{diagrams:generic} shows.
\termidx{CROW }generates each one of these variants with fine-grained diversification at instruction level, applying the majority of the strategies discussed in \autoref{sota:sota} and \emph{constant inferring}. Besides, \tool inlines function variants when appropriate, also resulting in call stack diversification at runtime.

In \autoref{workflow}, we summarize the analysis and transformation pipeline of \tool.
We pass a bitcode to be diversified, as an input to \tool.
% Move this to the previous section
%LLVM binaries can be obtained from any language with an LLVM frontend such as C/C++, Rust or Go, and they can easily be compiled to WebAssembly.
In Step~\step{1}, the binary is passed to CROW. 
Step~\step{2} packages all the variants in one single \termidx{multivariant }LLVM binary. 
In Step~\step{3}, we use a special component, called a ``mixer``,  which augments the binary with a random generator, which is required for performing the execution-path randomization. 
Also at this stage, the \termidx{multivariant }binary is fixed with the entrypoint of the original binary.
%The harness is used to connect the program to its original execution environment while the generator provides support for random execution path at runtime.
The final output of Step~\step{4} is a standalone \termidx{multivariant }\wasm binary that can be directly deployed. 
The source code of \termidx{MEWE }can be found at \todo{}.

%For sake of open science and for fostering research on this important topic, the code of \tool is made publicly available on GitHub: \repourl.

\begin{figure*}
  \centering
  \includegraphics[width=\linewidth]{diagrams/MEWE.pdf}
  \caption{Overview of \tool. It takes as input an LLVM binary. It first generates a set of functionally equivalent variants for each function in the binary and then generates a LLVM \termidx{multivariant }binary composed of all the function variants. Also, it includes the dispatcher functions in charge of selecting a variant when a function is invoked. The \tool mixer composes the LLVM \termidx{multivariant }binary with a random number generation library and a tampering of the original application entrypoint, in order to produce a \wasm \termidx{multivariant }binary ready to be deployed. }
  \label{workflow}
\end{figure*}

\begin{comment}

%\subsection{Variant generation}

% How \termidx{CROW }works ?
%\tool relies on the superdiversifier CROW \cite{CabreraArteaga2020CROWCD} to automatically diversify each function in the input LLVM binary (Step~\step{1}). \termidx{CROW }receives an LLVM module, analyzes the binary at the function block level and generates semantically equivalent variants for each function, if they exist. \termidx{CROW }variants are verified as semantically equivalent with an SMT solver. 
%Here, we define a function variant as:

%\begin{definition}{Function variant:}\label{def:variant}
%    Let $F$ be a function, $F'$ is a function variant of $F$ for \tool if it is semantically equivalent  (i.e., same input/output behavior), but exhibits a different internal behavior through tracing. 
%\end{definition}

%In \autoref{example:eq_code}, we illustrate two semantically equivalent Wasm functions according to \autoref{def:variant}. 
%The left most listing corresponds to the Wasm module shown in \autoref{WasmExample}.
%The right most listing is a variant for this function.
%We can appreciate that the multiplication of the original code, in the third and four lines, is replaced by an addition, making the variant to have the same semantic but executing different instructions.
\lstset{
    language=WAT,
    style=WATStyle,
    stepnumber=0,
    label=EQExample}
\begin{minipage}{0.9\linewidth}
    \begin{minipage}{0.4\linewidth}
        \begin{lstlisting}
...
(func (;0;)
    local.get 0
    local.get 0
    i32.const 2
    i32.mul
    i32.add)
...
        \end{lstlisting}
    \end{minipage}
        \begin{minipage}{0.45\linewidth}
        \begin{lstlisting}
...
(func (;0;)
    local.get 0
    local.get 0
    local.get 0
    i32.add
    i32.add)
...
        \end{lstlisting}
    \end{minipage}
    \noindent\rule{\linewidth}{0.4pt}
    \captionof{lstlisting}{Example of two semantically equivalent functions. The left listing corresponds to the original code. The right listing shows a semantically equivalent variant.}\label{example:eq_code}

\end{minipage}

\vspace{2mm}

\termidx{CROW }synthesizes variants by enumerative synthesis based on code transformation. 
The most relevant transformations are: constant inferring to replace control flow statements, arithmetic's equivalent replacement, and loop unrolling. 
\termidx{CROW }performs stacked transformations, this means that it can synthesize variants of different size, i.e., from smaller to larger variants than the original.

% Soundness
The variants created by \termidx{CROW }are artificially synthesized from the original binary. 
\termidx{CROW }checks for semantic equivalence of both codes, original and variant using the symbolic execution.
If the behavior of the variant is not the reference behavior, the variant is discarded.
This means that, after Step \step{1}, the variant is necessarily equivalent to the original program.


\end{comment}

\subsection*{Combining variants into \termidx{multivariant }functions}

The key component of \tool consists in combining the variants generated for the original functions, into a single binary.
The goal is to support execution-path randomization at runtime.
% General idea
The core idea is to introduce one dispatcher function per original function for which we generate variants.
A dispatcher function is a synthetic function that is in charge of choosing a variant at random, every time the original function is invoked during the execution.
With the introduction of dispatcher function,  \tool turns the original call graph into a \termidx{multivariant }call graph, defined as follows. 

\begin{definition}{Multivariant Call Graph (MCG):}\label{def:EP}
    A \termidx{multivariant }call graph is a call graph $\langle N,E \rangle$ where the nodes in $N$ represent all the functions in the binary and an edge $(f_1,f_2) \in E$ represents a possible invocation of $f_2$ by $f_1$  \cite{ryder1979}, where the nodes are typed. The nodes in $N$ have three possible types: a function present in the original program,  a generated function variant, or a dispatcher function.
\end{definition}


\begin{figure}
    \centering
  \includegraphics[width=.8\linewidth]{diagrams/CFG.pdf}
  \caption{Example of two static call graphs. At the top, the original call graph, at the bottom, the \termidx{multivariant }call graph, which includes nodes that represent function variants (in grey), dispatchers (in green), and original functions  (in yellow).
}
  \label{multivariant}
\end{figure}

% Instance of a \termidx{multivariant }module
In \autoref{multivariant}, we show the original static call graph for and original program (top of the figure), as well as the \termidx{multivariant }call graph generated with \tool (bottom of the figure).
The grey nodes represent function variants, the green nodes function dispatchers and the yellow nodes are the original functions.
The possible calls are represented by the directed edges.
The original program includes 3 functions. \tool generates 43 variants for the first function, none for the second and three for the third function. 
\tool introduces two dispatcher nodes, for the first and third functions. Each dispatcher is connected to the corresponding function variants, in order to invoke one variant randomly at runtime.


% exaplanation of dispatcher
In  \autoref{listing:multivariant_template}, we illustrate the LLVM construction for the function dispatcher corresponding to the right most green node of \autoref{multivariant}.
% General logic of a multivartiant function
It first calls the random generator, which returns a value that is then used to invoke a specific function variant. It should be noted that the dispatcher function is constructed using the same signature as the original function. 


\lstset{
    language=llvm,
    %style=nccode,
    basicstyle=\footnotesize\ttfamily,
    columns=fullflexible,
    breaklines=true,
    numbers=none,
    stepnumber=1,
    float
}

\begin{code}
\scriptsize
\noindent\begin{minipage}[b]{\linewidth}
    \begin{minipage}[t]{1\linewidth}
        \begin{lstlisting}[escapeinside={(*}{*)}]
define internal i32 @foo(i32 %0) {
    entry:
      %1 = call i32 @discriminate(i32 3)
      switch i32 %1, label %end [
        i32 0, label %case_43_
        i32 1, label %case_44_
      ]
    case_43_:                 
      %2 = call i32 @foo_43_(%0)
      ret i32 %2
    case_44_:                
      %3 = <body of foo_44_ inlined>
      ret i32 %3
    end:                                             
      %4 = call i32 @foo_original(%0)
      ret i32 %4
}
        \end{lstlisting}
    \end{minipage}%
    
    \noindent\rule{\linewidth}{0.4pt}
    \captionof{lstlisting}{Dispatcher function embedded in the \termidx{multivariant }binary of the original function in the rightmost green node in \autoref{multivariant}.}\label{listing:multivariant_template}
\end{minipage}
\end{code}

% Why a linear based switch
We implement the dispatchers with a switch-case structure to avoid indirect calls that can be susceptible to speculative execution based attacks \cite{Narayan2021Swivel}. 
The choice of a switch-case also avoids having multiple function definitions with the same signature, which could increase the attack surface in case the function signature is vulnerable \cite{johnson2021}.
This also allows \tool to inline function variants inside the dispatcher, instead of defining them again.
Here we trade security over performance, since dispatcher functions  that perform indirect calls, instead of a switch-case,  could improve the performance  of the dispatchers as indirect calls have constant time.

\begin{comment}

\subsection*{MEWE's Mixer}

The \tool mixer has four specific objectives: tamper the entrypoint of the application, link the LLVM \termidx{multivariant }binary, inject a random generator and merge all these components into a \termidx{multivariant }\wasm binary.
% Implementation
We use the Rustc compiler\footnote{\url{https://doc.rust-lang.org/rustc/what-is-rustc.html}} to orchestrate the mixing.
% The random number generation
For the random generator, we rely on WASI's specification \cite{WASI} for the random behavior of the dispatchers. Its exact implementation is dependent on the platform on which the binary is deployed.

The \tool mixer creates a new entrypoint for the binary called \emph{entrypoint tampering}.
It simply wraps the dispatcher for the entrypoint variants as a new function for the final Wasm binary and is declared as the application entrypoint. %This fixes an issue with  

\end{comment}
% How to turn function into endpoints
%The entrypoint tampering is needed for binariess passed to MEWE. We refer to the entrypoint as the entering main function of the original binary passed to MEWE. The tampering is needed because dispatchers for entrypoint variants make the \termidx{multivariant }invalid to be directly executed after it is compiled to Wasm.

%Throughout this paper, we refer to an endpoint as the closure of invoked functions when the entry point of the \wasm binary is executed.



\begin{comment}

\subsection{Multivariant Binary Execution at the Edge}

\todo{Introduce as the use of MEWE}

% What really is executed is the x86 code
When a WebAssembly binary is deployed on an edge platform, it is translated to machine code on the fly.
For our experiment, we deploy on the production edge nodes of Fastly. This edge computing platform uses Lucet, a native WebAssembly compiler and runtime, to compile and run the deployed Wasm binary \footnote{\url{https://github.com/bytecodealliance/lucet}}.
Lucet generates x86 machine code and ensures that the generated machine code executes inside a secure sandbox, controlling memory isolation.


\begin{figure}
\centering
  \includegraphics[width=1\linewidth]{diagrams/traces.pdf}
  \caption{Top: an execution trace for the  \texttt{bin2base64} endpoint. Middle and bottom: two different execution traces for the \termidx{multivariant }\texttt{bin2base64}, exhibited by two different requests with exactly the same input.}
  \label{http:workflow}
\end{figure}

% How it works
\autoref{http:workflow} illustrates  the runtime behavior of the original and the \termidx{multivariant }binary,  when deployed on an Edge node.
The top most diagram illustrates the execution trace for the  original of the endpoint \texttt{bin2base64}.
When the HTTP request with the input \texttt{"HelloWorld!"} is received, it invokes functions $f1$, $f2$ followed by 27 recursive calls of function $f3$. Then, the endpoint sends the result \texttt{"0x000xccv0x10x00b3Jsx130x000x00 0x00xpopAHRvdGE="} of its base64 encoding in an HTTP response.

The two diagrams at the bottom of \autoref{http:workflow} illustrate two executions traces observed through two different requests to the endpoint \texttt{bin2base64}.
In the first case, the request first triggers the invocation of dispatcher $d1$, which randomly decides to invoke the variant $f1_2$; then $f2$, which has not been diversified by \tool, is invoked; then the recursive invocations to $f3$ are replaced by iterations over the execution of dispatcher $d2$ followed by a random choice of variants of $f3$. Eventually the result is computed and sent back as an HTTP response. 
The second execution trace of the \termidx{multivariant }binary shows the same sequence of dispatcher and function calls as the previous trace, and also shows that for a different requests, the variants of $f1$ and $f3$ are different. 


The key insights from these figures are as follows. First, from a client's point of view, a request to the original or to a \termidx{multivariant }endpoint, is completely transparent. Clients send the same data, receive the same result, through the same protocol, in both cases.
Second, this figure shows that, at runtime, the execution paths for the same endpoint are different from one execution to another, and that this randomization process results from multiple random choices among function variants, made through the execution of the endpoint.
%From an attacker's perspective, this random selection of variants constantly moves the attack surface and is meant to render a potential vulnerability harder to reach or exploit \cite{davi2015isomeron, 10.5555/3091125.3091155, BEKIROGLU2021106601}.


\end{comment}



\section*{Accompanying Source Code}

This thesis is accompanied by the source code of both contributions, CROW and MEWE. The source code is accessible through the links:
\begin{enumerate}
    \item CROW: \todo{slumps}
    \item MEWE: \todo{mewe}
\end{enumerate}

Our software artifacts are licensed under the MIT License. The dependent source codes, such as LLVM, are licensed under their original licenses.

\section*{Conclusions}

This chapter discusses the technical details for the artifacts implemented for our two contributions.
We describe how CROW generates program variants.
We introduce a new mutation strategy that is a consequence of retargeting a superoptimizer for LLVM as a superdiversifier.
Besides, we dissect MEWE and how it creates an MVE system.
In \autoref{chapter:method} we discuss the methodology we follow to evaluate the impact of CROW and MEWE.
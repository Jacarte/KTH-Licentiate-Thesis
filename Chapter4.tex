\chapter{Results} 

In this chapter, we sum up the results of the research of this thesis.  Instead, we illustrate the key insights and challenges faced in answering each research question.  To obtain our results, we followed the methodology formulated in \autoref{chapter:method}.


\section{\rqone}
\label{results:rq1}

\newcommand{\diversifiedsodium}{85}
\newcommand{\diversifiedqrcode}{32}
\newcommand{\libsodiumpopulation}{4272}
\newcommand{\qrpopulation}{6369}


\newcommand{\allmewediversified}{\diversifiedsodium + \diversifiedqrcode}
\newcommand{\allmewepopulation}{\libpopulation + \qrpopulation}

As we describe in \autoref{rq1:method}, our first research question aims to answer how to generate \wasm program variants. \todo{Motivate} We pass each function of the corpora listed in \autoref{table:corpora} to CROW, and we collect how many variants CROW generates for each function.
This section is organized as follows. First we present the general results for \emph{Population size}(\autoref{metric:md5sum}) for each corpus. 
Second, we discuss the challenges and limitations in program variants generation. Finally, we illustrate the most common code transformations performed by our approach and answer RQ1.

\subsection*{Program's populations}

We summarize the results in \autoref{table:crow:general_results}.
The table is illustrates the Corpus name, the number of functions to diversify, the number of successfully diversified functions (functions with at least one artificially created variant) along with the percentage of successfully diversified functions, the cumulative number of variants taking into account all programs in the corpus and the relation between the increased population size and the original number of functions in the corpus.

% General results on the number of diversified functions per corpus
We produce at least one unique program variant for $239/303{}$ single function programs for \corpusrosetta with one hour for a timeout. For the rest of the programs ($64/303{}$), the timeout is reached before CROW can find any valid variant. 
In the case of \corpussodium and \corpusqrcode, we produce variants for $\py{\diversifiedsodium}/\py{\libsodiumfunctions}$ and $\py{\diversifiedqrcode}/\py{\qrcodefunctions}$ functions respectively, with 5 minutes per function as timeout. The rest of the functions resulted in timeout before finding function variants or produce no variants. For all programs in all corpora, we achieve $356/3021$ successfully diversified functions, for a $11.78\%$.
% General growing factor
As the four and fifth columns show, the number of artifically created variants increased the original population $4.15$ times, from $3021$ programs to $12547$.


{
    \renewcommand{\arraystretch}{1.6}
\begin{table}[h]
    %\small
    \centering
    %\setlength\minrowclearance{1.0pt}
        \begin{tabular}[t]{ l | r | l  | l | r}
     Corpus & \#Functions & \# Diversified &  \# Variants & \#  Unique  Variants \\
        \hline   

        \corpusrosetta & \fromjson{data/crow_corpus.json}{.[0].functions} & 239 & \rossetapopulation  &  \rossetapopulationunique \\
        \hline


        \corpussodium & \libsodiumfunctions & \diversifiedsodium & \libsodiumpopulation & \libsodiumpopulationunique     \\
        \hline

        \corpusqrcode & \qrcodefunctions & \diversifiedqrcode  &  \qrpopulation & \qrpopulationunique \\

        \hline\hline

         & \py{\libsodiumfunctions + \qrcodefunctions + 303 } & \py{239 + \diversifiedqrcode +\diversifiedsodium}  & \py{\qrpopulation + \rossetapopulation + \libsodiumpopulation} & \py{\qrpopulationunique + \rossetapopulationunique + \libsodiumpopulationunique}   \\
        \end{tabular}
    
        \caption{General program's populations statistics. The table is composed by the name of the corpus, the number of functions, the number of succesfully diversified functions, the cumulative number of generated variants and the cumulative number of unique variants.}
        \label{table:crow:general_results}
\end{table}
}


\subsection*{Challenges for automatic diversification}
\label{rq1:challenges}

We have observed a remarkable difference between the number of successfully diversified functions versus the number of failed-to-diversify functions (third column of \autoref{table:crow:general_results}). Our approach successfully diversified approx. 79 \%, \py{100*{\diversifiedsodium} / {\libsodiumfunctions}}\% and \py{100*{\diversifiedqrcode} / {\qrcodefunctions}}\% of the original functions for \corpusrosetta, \corpussodium and \corpusqrcode respectively. 

Setting up the timeout affects the capacity for diversification. A low timeout for exploration gives our approach more power to combine code replacements. We can appreciate this in the last column of the table, where for a lower number of diversified functions, we create, overall, more variants.

%Moreover, we look at the cases that yield a few variants per function. There is no direct correlation between the number of identified codes for replacement and the number of unique variants. Therefore, 

Apart from the timeout, we manually analyze programs searching for properties attempting to the generation of program variants using CROW.
We identify two main challenges for diversification.

\emph{1) Constant computation}  We have observed that our approach searches for a constant replacement for more than $45\%$ of the blocks of each function while constant values cannot be inferred. For instance, constant values cannot be inferred for memory load operations because our tool is oblivious to a memory model. 

\emph{2) Combination computation}  The overlap between code replacements, is a second factor that limits the number of unique variants. We can generate a high number of variants, but not all replacement combinations are necessarily unique.

%\todo{Add all the found examples here}


\subsection*{Properties for large diversification}

We manually analyzed the programs that yield more than 100 variants to study the critical properties of programs producing a high number of variants.
This reveals one key factor that favors many unique variants: the presence of bounded loops. In these cases, we synthesize variants for the loops by replacing them with a constant, if the constant inferring \cite{souper} is successful. Every time a loop constant is inferred, the loop body is replaced by a single instruction. This creates a new, statically different program. The number of variants grows exponentially if the function contains nested loops for which we can successfully infer constants. 

% 
\todo{Below is cryptic, take a look}
The before mentioned factor seems to be contradictory to the challenges mentioned in \autoref{rq1:challenges}. Constant inferring could generate program variants as it can attempt against their generation. In previous studies \citationneeded, the authors highlight that more than 40\% of code can be replaced with a constant. 

A second key factor for synthesizing many variants relates to the presence of arithmetic. The synthesis engine used by our approach, effectively replaces arithmetic instructions with equivalent instructions that lead to the same result. For example, we generate unique variants by replacing multiplications with additions or shift left instructions (\autoref{add:example}). Also, logical comparisons are replaced, inverting the operation and the operands (\autoref{cmp:examples}). Besides, our implementation can use overflow and underflow of integers to produce variants (\autoref{overflow:example}), using the intrinsics of the underlying computation model.

{
\begin{code}
    \footnotesize
    \lstdefinestyle{nccode}{
        numbers=none,
        firstnumber=2,
        stepnumber=1,
        numbersep=10pt,
        tabsize=4, 
        showspaces=false,
        breaklines=true, 
        showstringspaces=false,
        moredelim=**[is][\btHL]{`}{`},
        moredelim=**[is][{\btHL[fill=black!10]}]{`}{`},
        moredelim=**[is][{\btHL[fill=celadon!40]}]{!}{!}
    }

    \lstset{
        language=WAT,
        style=nccode,
        basicstyle=\footnotesize\ttfamily,
        columns=fullflexible,
        breaklines=true
    }
    \noindent\begin{minipage}[b]{0.32\linewidth}
        \captionof{lstlisting}{Diversification through arithmetic expression replacement.}\label{add:example}
        \noindent\begin{minipage}[t]{0.46\linewidth}
            \begin{lstlisting}
local.get 0
`i32.const 2`
`i32.mul`
            \end{lstlisting}
        \end{minipage}%
        \hfill\noindent\begin{minipage}[t]{0.46\linewidth}
            
            \begin{lstlisting}
local.get 0
!i32.const 1!
!i32.shl!
            \end{lstlisting}
        \end{minipage}
    \end{minipage}\hfill%
    \begin{minipage}[b]{0.31\linewidth}
        \captionof{lstlisting}{Diversification through inversion of comparison operations.}\label{cmp:examples}
        \begin{minipage}[t]{.46\linewidth}
            \begin{lstlisting}
`local.get 0`
`i32.const 10`
`i32.gt_s`
            \end{lstlisting}
        \end{minipage}\hfill\begin{minipage}[t]{.46\linewidth}
           
            \begin{lstlisting}
!i32.const 11!
!local.get 0!
!i32.le_s!
            \end{lstlisting}
        \end{minipage}%
        
        
    \end{minipage}\hfill\noindent
    \noindent\begin{minipage}[b]{0.32\linewidth}
        \captionof{lstlisting}{Diversification through overflow of integer operands.}\label{overflow:example}
        \noindent\begin{minipage}[t]{0.46\linewidth}
            \begin{lstlisting}
`i32.const 2`
i32.mul
\end{lstlisting}
        \end{minipage}%
        \hfill\noindent\begin{minipage}[t]{0.46\linewidth}
            
            \begin{lstlisting}
i32.const 2
i32.mul
!i32.const -2147483647!
!i32.mul!
            \end{lstlisting}
        \end{minipage}
    \end{minipage}
    \end{code}
}



\section{Answer to RQ1.}

We can provide diversification for 11.78\% of the programs in our corpora. We increase the initial count of programs by a factor of $4.15$. We identify the challenges attempting against the automated creation of programs variants, \emph{Constant computation} and \emph{Combination computation}. Nevertheless, the same constant computation, complemented with the high presence of arithmetic operations in the original program increased the number of program variants. 


\begin{comment}

While our work is very limited by the provided corpora, it can be easily extended to other\dots

An application that benefits from the ablity of CROW is the large amount of generated variants. 


- CHeckLong questions for 80%
- Stress how to do experiments in Software Engineering, the theory behind how to do this in CS.
- The objects of the experiments are programs that ... 
- Motivate the corpora selection, size, can be ported, security sensitive, etc,
- Do not mention CROW in the selection criteria.
- The motivation is not related to tools, only to concepts
- Move paragraph after listing of coprpora to before.
- 
%The proposed methodology can generate program variants that are syntactically different from their original versions. We have shown that CROW generates diversity among the binary code variants using semantically equivalent code transformations. We identified the properties that original programs should have to provide a handful number of variants. Besides, we enumerated the challenges faced to provide automatic diversification by retargeting a superoptimizer.

%In the next chapter, we evaluate the assessment of the generated variants answering to what extent the artificial programs are different from the original in terms of static difference, execution behavior, and preservation.

\end{comment}

\section{\rqtwo}



Our second research question investigates the differences between program variants at runtime.
To answer RQ2, we execute each program/variant to collect their execution traces and execution times.
For each programs' population we compare \autoref{metric:stack} and \autoref{metric:time} for each program/variant pair.
This section is organized as follows. First, we analyze the programs' populations by comparing the values of \autoref{metric:stack} for each pair of program/variant. The pairwise comparison will hint at the results at the population level. We want to analyze not only the differences of a variant regarding its original program, but we also want to compare the variants against other variants. Second, we do the same pairwise strategy for \autoref{metric:time}, performing a Mann-Withney U test for each pair of program/variant times distribution. Finally, we conclude and answer RQ2.

\subsection*{Stack operation traces.}

% Describe the first figure
In \autoref{rq2:dtw_distrib} we plot the distribution of all \DTW comparisons for all pairs of program/variant of each program's population generated un RQ1. All compared programs are statically different. Each vertical group of dots (in logarithmic scale) represents the \DTW values for a program of the \corpusrosetta corpus for which we generated variants. 

%We want to remark that the programs excluded from the plot all result from their comparisons in \DTW larger than zero. Thus, the unique generated variants present a different behavior in terms of stack operation traces.


\begin{figure}[h]
    \centering
    \includegraphics[width=\linewidth]{plots/plot_distribs1.png}
    \includegraphics[width=\linewidth]{plots/plot_distribs2.png}
    \includegraphics[width=\linewidth]{plots/plot_distribs3.png}
    \caption{Pairwise of \autoref{metric:stack} values in logarithmic scale. Each vertical plot represents a program and its variants. The plot only contains the programs for which we generate more than one variant, \ie more than one pair of programs comparisons. }
    \label{rq2:dtw_distrib}
\end{figure}

% Main insight
We have observed that in the majority of the cases, the mean of the comparison values is huge in all cases. We analyze the length of the traces, and one reason behind such large values of \DTW is that some variants result from constant inferring. For example, if a loop is replaced by a constant, instead of several symbols in the stack operation trace, we observe one. Consequently, the distance between two program traces is significant. We have observed no relation between how aggressive the transformation is and the length of the traces.  

In some cases, even when the variants are statically different, two programs/variants result in a \DTW value oz zero, \ie they result in the same stack operation trace. We identified two main reasons behind this phenomenon. First, the code transformation that generates the variant targets a non-executed or dead code. This result calls for future work on correct code debloating \citationneeded. Second, some variants have two different instructions that trigger the same stack operations. For example, the code replacements below illustrate the case. The four cases leave the same value in the stack operation trace.

\begin{code}[H]
\centering
\noindent\begin{minipage}{.23\linewidth}
\lstdefinestyle{nccode2}{
    numbers=none,
    firstnumber=1,
    stepnumber=1,
    numbersep=10pt,
    tabsize=4,
    showspaces=false,
    breaklines=true, 
    showstringspaces=false,
    moredelim=**[is][{\btHL[fill=black!10]}]{`}{`},
    moredelim=**[is][{\btHL[fill=celadon!40]}]{!}{!}
}
    \lstset{
        language=WAT,
        style=nccode2,
        basicstyle=\footnotesize\ttfamily,
        columns=fullflexible,
        breaklines=true
    }
    \begin{lstlisting}
(1) `i32.lt_u`
(2) `i32.le_s`
    \end{lstlisting}
\end{minipage}\hfill%
\noindent\begin{minipage}{0.2\linewidth}
\lstdefinestyle{nccode2}{
    numbers=none,
    firstnumber=1,
    stepnumber=1,
    numbersep=10pt,
    tabsize=4,
    showspaces=false,
    breaklines=true, 
    showstringspaces=false,
    moredelim=**[is][{\btHL[fill=black!10]}]{`}{`},
    moredelim=**[is][{\btHL[fill=celadon!40]}]{!}{!}
}

    \lstset{
        language=WAT,
        style=nccode2,
        basicstyle=\footnotesize\ttfamily,
        columns=fullflexible,
        breaklines=true
    }
    \begin{lstlisting}
!i32.lt_s!
!i32.lt_u!
    \end{lstlisting}
\end{minipage}\hfill%
\noindent\begin{minipage}{.3\linewidth}
\lstdefinestyle{nccode2}{
    numbers=none,
    firstnumber=1,
    stepnumber=1,
    numbersep=10pt,
    tabsize=4,
    showspaces=false,
    breaklines=true, 
    showstringspaces=false,
    moredelim=**[is][{\btHL[fill=black!10]}]{`}{`},
    moredelim=**[is][{\btHL[fill=celadon!40]}]{!}{!}
}

    \lstset{
        language=WAT,
        style=nccode2,
        basicstyle=\footnotesize\ttfamily,
        columns=fullflexible,
        breaklines=true
    }
    \begin{lstlisting}
(3) `i32.ne`
(4) `local.get 6`
    \end{lstlisting}
\end{minipage}\hfill%
\noindent\begin{minipage}{0.2\linewidth}
\lstdefinestyle{nccode2}{
    numbers=none,
    firstnumber=1,
    stepnumber=1,
    numbersep=10pt,
    tabsize=4,
    showspaces=false,
    breaklines=true, 
    showstringspaces=false,
    moredelim=**[is][{\btHL[fill=black!10]}]{`}{`},
    moredelim=**[is][{\btHL[fill=celadon!40]}]{!}{!}
}

    \lstset{
        language=WAT,
        style=nccode2,
        basicstyle=\footnotesize\ttfamily,
        columns=fullflexible,
        breaklines=true
    }
    \begin{lstlisting}
!i32.lt_u!
!local.get 4!
    \end{lstlisting}
\end{minipage}
\end{code}



In \autoref{rq2:zero_ratio} we plot those programs for which at least one comparison (\DTW) is zero. The plot contains 82 vertical bars, one for each program. In the other cases, all \DTW values are non-zero. We can observe that even when some programs/variants result in the same trace, there is no case for which all variants return the same stack operation trace with all bars above $Y=0.2$. There is always at least one generated variant that is dynamically different from the original program. 

\newcommand{\zerocountprogs}{82\xspace}

\subsection*{Execution times.}

% Overall results
For each program's population, we compare the execution time distributions, \autoref{metric:time}, of each pair of program/variant.
Overall diversified programs, 169 out of 239 have at least one variant with a different execution time distribution than the original program (P-value $<$ $0.01$ in the Mann-Withney test). This result shows that we effectively generate variants that yield significantly different execution times.

By analyzing the data, we observe the following trends. First, if our tool infers control-flows as constants in the original program, the variants execute faster than the original, sometimes by one order of magnitude. On the other hand, if the code is augmented with more instructions, the variants tend to run slower than the original. 

In both cases, we generate a variant with a different execution time than the original. Both cases are good from a randomization perspective since this minimizes the certainty a malicious user can have about the program's behavior. Therefore, a deeper analysis of how this phenomenon can be used to enforce security will be discussed in answering RQ3.

To better illustrate the differences between executions times in the variants, we dissect the execution time distributions for two programs' populations. 
% describe figures


\begin{figure*}[h]
    \centering
    \includegraphics[width=\linewidth]{plots/hilbert_curve.pdf}
    \caption{Execution time distributions for \texttt{Base64\_decode} and \texttt{Hilbert\_curve} program and their variants in top and bottom figures respectively. Baseline execution time mean is highlighted with the magenta horiontal line. }
    \label{rq3:perf}
\end{figure*}

The plots in \autoref{rq3:perf} show the execution time distributions of programs \texttt{Base64\_decode} and \texttt{Hilbert\_curve} and their variants. 
We illustrate time diversification with these two programs because, for both, we generate unique variants with all types of transformations previously discussed in \autoref{results:rq1}.
In the plots along the X-axis, each vertical set of points represents the distribution of 100 execution times per program/variant. The Y-axis represents the execution time value in milliseconds. The original program is highlighted in magenta color: the distribution of 100 execution times is given on the left-most part of the plot, and its median execution time is represented as a horizontal dashed line. The median execution time is represented as a blue dot for each execution time distribution, and the vertical gray lines represent the entire distribution. The bolder gray line represents the 75\% interquartile. The program variants are sorted concerning the median execution time in descending order.



% explanation
For \texttt{Base64\_decode}, the majority of variants are constantly slower than the reference programs (blue dot above the magenta line). For \texttt{Hilbert\_curve}, many diversified variants are optimizations (blue dots below the magenta bar). The case of \texttt{Hilbert\_curve} is graphically clear, and the last third represents faster variants resulting from code transformations that optimize the original program.
Our tool provides program variants in the whole spectrum of time executions, lower and faster variants than the original program. The developer is in charge of deciding the trade-off between taking all variants or only the ones providing the same or less execution time for the sake of less overhead. 

\section{Answer to RQ2.}
We empirically demonstrate that our approach generates program variants for which their execution traces are different. We stress the importance of complementing static and dynamic studies of programs variants. For example, if two programs are statically different, that does not necessarily mean different runtime behavior. There is at least one generated variant for all executed programs that provides a different execution trace. 
% answer to RQ2
We generate variants that exhibit a significant diversity of execution times. For example, for $169/239\,(71\%)$ of the diversified programs, at least one variant has an execution time distribution that is different compared to the execution time distribution of the original program. 
The result from this study encourages the composition of the variants to provide a more resilient execution.

\section{\rqthree}



\section{Answer to RQ3.}
% Define some numbers here for the autmation of the tables

\pagebreak
\section{Conclusions}

This work proposes and evaluates our approach to generate \wasm program variants artificially. Our approach generates different, both statically and dynamically, variants for up to 70\% of the programs in our three corpora. While generating statically different programs is still important, we highlighted the importance of complementing static and dynamic studies for programs diversification. We generate variants that exhibit a significant diversity of execution times that we can use to compose time-unpredictable multivariant binaries. Our approach of combining function variants in a single function as multivariants successfully triggers diverse execution paths and execution times at runtime. Therefore, making it virtually impossible for an attacker to predict which code is executed for a given query. Finally, we demonstrate the feasibility of automatically generating \wasm program variants.

% While our approach targets \wasm environments, previous works have demonstrated the importance of execution path randomization \cite{davi2015isomeron}. The work of Davi \etal proposed to randomly switch between two program executions, the original and a variant to provide an unpredictable behavior. 

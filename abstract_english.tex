\begin{abstract}



% Birth
\wasm has become the fourth official web language. 
This new language allows web browsers to execute existing programs or libraries written in other languages, such as C/C++ and Rust.
Apart from web browsers, \wasm evolves to be part of Edge-Cloud computing platforms. 
% Problems
Despite being designed with security as a premise, it is not exempt from vulnerabilities. Our approaches deal with this fact by providing a preemptive solution with software diversification.

In this thesis, we propose an automatic approach to generate software diversification for \wasm programs. 
In addition, we provide complementary implementation for our approaches, including a generic LLVM superdiversifier that potentially extends our ideas to other programming languages.
We empirically demonstrate the impact of our approach by providing Randomization and Multivariant Execution (MVE) for \wasm. 
Our results show that our approaches can provide an automated end-to-end solution for the diversification of \wasm programs. 
The main contributions of this work are:

\begin{itemize}


    \item We highlight the lack of diversification techniques for WebAssembly through an exhaustive literature review.
    
    \item We provide the implementation of two tools, CROW and MEWE. These tools provide randomization and multivariant execution for \was respectively. 


    \item We include \emph{constant inferring} as a new code transformation to generate software diversification for \wasm.

    \item We empirically demonstrate the impact of our technique by evaluating the static and dynamic behavior of the generated diversification.
    
\end{itemize}
 
Our approaches harden observable properties commonly used to conduct attacks, such as static code analysis, execution traces, and execution time.
Therefore, our approaches harden unknown and yet-unknown vulnerabilities.

\textbf{Keywords:} WebAssembly, Software Diversification, Automatic Software Engineering, Security 
\\
\\
\\
\\
\clearpage
\end{abstract}
\endinput

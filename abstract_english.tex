\begin{abstract}

    \begin{comment}


    % Birth
The Web Consortium (W3C) standarized bytecode for the web environment with the \wasm (Wasm) language in 2015. 
% Evolution, history and importance
Wasm allows web browsers to execute existing programs or libraries written in other languages, such as C/C++ and Rust.
Beyond web environments, \wasm evolves to be part of Edge-Cloud computing platforms \cite{9640153, wen2020wasmachine}. 
% Problems
Despite being designed for sandboxing and secure execution, it is not exempt from vulnerabilities \cite{WebAssemblySecurity}.
% Problem instantiation
For example, \wasm engines are vulnerable to speculative execution \cite{Narayan2021Swivel}, and C/C++ source code vulnerabilities might be ported to Wasm binaries \cite{DeRoover2022}.  

% MTD and the need for variants
One strategy to hide such vulnerabilities is to move them in time as a preemptive solution.
The goal is to make potential vulnerabilities available only in a time window. This makes potential attackers not hit what they cannot see.  
This strategy is usually called Moving Target Defense (MTD) \cite{MTDNationalCyberLaep, okhravi2013survey}. 
MTD for software is a collection of techniques that aim to improve the security of a system by constantly rotating its vulnerable programs from one variant to another. 
A program variant should be different from the original program but functionally equivalent to it.
By rotating the deployment and execution between the program variants, a potential attacker needs more efforts to perform the same attack for all variants \cite{sengupta}.
Thus, one premise for effectively implementing MTD for a given program is the need for the program variants.


% SD
In MTD, Software Diversification is the process of finding, creating, and deploying program variants.
Usually, program variants could be found in the wild in a phenomenon called natural diversity \cite{Harrand1650630}. In the case of WebAssembly, since it is a novel technology, there is no natural diversity. Thus, effective MTD cannot be implemented due to the lack of program variants.
This work proposes to create program variants for \wasm artificially. 
Therefore, we aim to generate artificial software diversification for WebAssembly.
To reach such a goal, we answer three research questions enunciated in the following.



\wasm has become a new technology for web browsers and standalone engines such as the ones used in Edge-Cloud platforms. \wasm is designed with security and sandboxing premises, yet, is still vulnerable.
Besides, since it is a relatively new technology, new vulnerabilities appear in the wild faster than the adoption of patches and defenses.
As a widely studied field, software diversification could be a solution for known and yet-unknown vulnerabilities. Yet, there is no research on this field for \wasm.

We propose an automatic approach to generate software diversification for \wasm in this work. 
In addition, we provide complementary implementation for our approaches, including a generic LLVM superdiversifier that potentially extends our ideas to other programming languages.
We empirically demonstrate the impact of our approach by providing Randomization and Multivariant Execution (MVE) for \wasm.
For this, we provide two tools, CROW and MEWE. CROW completely automatizes the process by using a superdiversifier. Besides, MEWE provides execution path randomization for an MVE.
This chapter is organized into two sections. 
In \autoref{conclusions:summary}, we summarize the main results we found by answering our research questions enunciated in \autoref{chapter:intro}.
Finally, \autoref{future_work} describes potential future work that could extend this dissertation.

\section{Summary of the results}
\label{conclusions:summary}

We enunciate the three research questions in \autoref{chapter:intro}. 
With the first research question, we investigate the static properties of the software diversification for \wasm generated by our approaches. 
We answer our first research question by creating programs variants for \pypy{303 + \libsodiumfunctions + \qrcodefunctions} original programs. 
With CROW, we create program variants for the 11.78\% of the programs in our corpora.
We study the properties of the generated variants at the level of generated programs' population.
Thus, we identify the challenges that attempt against the generation of unique program variants.
Besides, we highlight the code properties that offer numerous program variants. 

Complementary with our first research question, we evaluate the dynamic properties of the program variants generated to answer our first research question.
We execute each of the 303 original programs and its generated variants for the \corpusrosetta.
For each execution, we collect their execution trace and their execution times.
We demonstrate that the \wasm variants generated by CROW offer remarkably different execution traces.
Similarly, the execution times are different between each program and its variants.
For the $71\%$ of the diversified programs, at least one variant has an execution time distribution that is different from the original program's execution time distribution.
Moreover, CROW generates both faster and slower variants.
Nevertheless, we highlighted the importance of dynamic analysis for software diversification. 

Our last and third research question evaluates the impact of providing a worldwide MVE for \wasm.
We use MEWE to build multivariant binaries for the program variants generated for \corpussodium and \corpusqrcode corpora.
We deploy the generated multivariant binaries in an Edge-Cloud platform, collecting their execution times.
The addition of runtime path randomization to multivariant binaries provides significant differences between the execution of the original binary and the multivariant binary.
The observed differences lead us to conclude that no specific variant can be inferred from studying the execution time of the multivariant binaries. Therefore, attacks that rely on measuring precise execution times are more challenging to conduct.


Overall, these results show that our approaches can provide an automated end-to-end solution for the diversification of \wasm programs. 
Our approaches harden observable properties commonly used to conduct attacks, such as static code analysis, execution traces, and execution time.
Therefore, our approaches harden unknown and yet-unknown vulnerabilities.


    \end{comment}
\textbf{Keywords:} WebAssembly, Software Diversification, Security 
\\
\\
\\
\\
\clearpage
\end{abstract}
\endinput

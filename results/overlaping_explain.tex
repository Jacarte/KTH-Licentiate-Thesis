Let us illustrate the case with the code in \autoref{CExample}. The \texttt{f} function calculates the value of $2 * x + x$ where \texttt{x} is the input for the function.  CROW compiles this source code and generates the intermediate LLVM bitcode in the left most part of \autoref{example:crow:original:llvm}. CROW potentially finds two code blocks to look for variants, as the right-most part of \autoref{example:crow:original:llvm} shows.

% snippet of code showing the detection of code blocks
    
\begin{code}
    \lstset{language=C,
    basicstyle=\small\ttfamily,caption={C function that calculates the quantity $2x + x$},label=CExample}
    \begin{lstlisting}[style=CStyle]
int f(int x) { 
    return 2 * x + x; 
}    
    \end{lstlisting}
    
\end{code}

\lstdefinelanguage{LLVM}
    {morekeywords={i32,mul,align,nsw,add,load,store,define,br, ret, shl, ret},
    sensitive=false,
    morecomment=[l]{;},
    morecomment=[s]{;}{;},
    morestring=[b],
}
\lstdefinestyle{nccode}{
    numbers=left,
    tabsize=4,
    showspaces=false,
    breaklines=true, 
    showstringspaces=false,
    moredelim=**[is][{\btHL[fill=black!10]}]{`}{`},
    moredelim=**[is][{\btHL[fill=celadon!40]}]{!}{!}
}
\lstset{
    language=LLVM,
    style=nccode,
    %basicstyle=\small\ttfamily,
    columns=fullflexible,
    breaklines=true
}


\begin{code}
    \centering
    \captionof{lstlisting}{LLVM's intermediate representation program, its extracted instructions and replacement candidates. Gray highlighted lines represent original code, green for code replacements. }\label{example:crow:original:llvm}
    \lstset{numbers=none}
    \noindent\begin{minipage}[t]{.33\linewidth}
    \centering
    \begin{lstlisting}[xleftmargin=1em,escapechar=?]
    define i32 @f(i32) {

    ?\tikzmarkWS{2}{code 2}{11.5}{10}{3.5cm}?
    ?\tikzmarkWS{1}{code 1}{11.5}{3.5}{3.0cm}?
    %2 = mul nsw i32 %0,2
    %3 = add nsw i32 %0,%2 

    ret i32 %3
    }
    
    define i32 @main() {
    %1 = tail call i32 @f(i32 10)
    ret i32 %1
    }
    \end{lstlisting}
    \end{minipage}%\hfill%
    \begin{minipage}[t]{.32\linewidth}
        \begin{lstlisting}[xleftmargin=1em,escapechar=?]
?Replacement candidates for code\_1?

`%2 = mul nsw i32 %0,2`

!%2 = add nsw i32 %0,%0!

!%2 = shl nsw i32 %0, 1:i32!
    \end{lstlisting}
    \end{minipage}%\hfill%
    \begin{minipage}[t]{.32\linewidth}
        \lstdefinestyle{nccode}{
        tabsize=4, 
        showspaces=false,
        breaklines=true, 
        showstringspaces=false,
        moredelim=**[is][{\btHL[fill=black!10]}]{`}{`},
        moredelim=**[is][{\btHL[fill=celadon!40]}]{!}{!}
        }
        \lstset{
            language=LLVM,
            style=nccode,
            columns=fullflexible,
            breaklines=true,
            belowcaptionskip=1pt,
            abovecaptionskip=1pt,
        } 
        \begin{lstlisting}[name={B},escapechar=?]
?Replacement candidates for code\_2?

`%3 = add nsw i32 %0,%2`

!%3 = mul nsw %0, 3:i32!
        \end{lstlisting}
    \end{minipage}
    
\end{code}





\begin{code}
    \centering
    \captionof{lstlisting}{Candidate code replacements combination. Orange highlighted code illustrate replacement candidate overlapping.}\label{example:crow:original:combination}
    \lstset{numbers=none}
    \noindent\begin{minipage}[t]{.5\linewidth}
    \begin{lstlisting}[xleftmargin=1em,escapechar=?]
`%2 = mul nsw i32 %0,2`
`%3 = add nsw i32 %0,%2`

!%2 = add nsw i32 %0,%0!
`%3 = add nsw i32 %0,%2`

!%2 = shl nsw i32 %0, 1:i32!
`%3 = add nsw i32 %0,%2`

    \end{lstlisting}
    \end{minipage}%\hfill%
    \begin{minipage}[t]{.5\linewidth}
        \lstdefinestyle{nccode}{
        tabsize=4, 
        showspaces=false,
        breaklines=true, 
        showstringspaces=false,
        moredelim=**[is][{\btHL[fill=black!10]}]{`}{`},
        moredelim=**[is][{\btHL[fill=celadon!40]}]{!}{!},
        moredelim=**[is][{\btHL[fill=weborange!40]}]{'}{'}
        }
        \lstset{
            language=LLVM,
            style=nccode,
            columns=fullflexible,
            breaklines=true,
            belowcaptionskip=1pt,
            abovecaptionskip=1pt,
        } 
        \begin{lstlisting}[xleftmargin=1em,escapechar=?]
'%2 = mul nsw i32 %0,2'
!%3 = mul nsw %0, 3:i32!

'%2 = add nsw i32 %0,%0'
!%3 = mul nsw %0, 3:i32!

'%2 = shl nsw i32 %0, 1:i32'
!%3 = mul nsw %0, 3:i32!

    \end{lstlisting}
    \end{minipage}
\end{code}


\begin{tikzpicture}[remember picture,overlay]
%\path (2.north) edge[<-, bend left] (1.north);
%\path[draw, ->] (3.west) edge[<-, bend left] (2.west);
%\path (4.west) edge[<-, bend left] (3.west);
%\path (1.south) edge[<-, bend left] (4.south);

%\path (2.east) edge[<-, bend left, blue] (5.north);
%\path (3.east) edge[<-, bend right, olive] (2.east);
%\path (1.east) edge[<-, bend left, black] (replall1.west);
%\path (2.east) edge[<-, bend left, black] (replall2.west);
%\path (rep11.east) edge[<-, bend left, black] (6.east);
%\path (9.east) edge[<-, bend right, black] (4.east);
%\path (7.east) edge[<-, bend right, black] (8.east);
%\path (5.south) edge[<-, bend right, blue] (4.east);
%\path (9.north) edge[<-] (8.south);
%\path (5.south) edge[<-, bend left] (9.south);


%\path (10.north) edge[<-, bend left] (11.north);
%\path (11.south) edge[<-, bend left] (10.south);
%\path (7) edge[<-, bend right] (6.east);
%\path (8) edge[<-, bend right] (7.east);
\end{tikzpicture}



CROW detects 2 code blocks, \texttt{code\_block\_1} and \texttt{code\_block\_2} as the enclosing boxes in the left most part of \autoref{example:crow:original:llvm} show. Hypothetically, CROW synthesizes $2 + 1$ candidate code replacements for each code block respectively as the green highlighted lines show in the right most parts of \autoref{example:crow:original:llvm}.
The baseline strategy of CROW is to generate variants out of all possible combinations of the candidate code replacements, \ie uses the power set of all candidate code replacements.

In the example, the power set is the cartesian product of the found candidate code replacements for each code block, including the original ones, as \autoref{example:crow:original:combination} shows. The power set size results in $6$ potential function variants. Yet, the generation stage would eventually generate $4$ variants from the original program. CROW generated 4 statically different Wasm  files, as \autoref{example:crow:variants:wasm} illustrates. This gap between the potential and the actual number of variants is a consequence of the redundancy among the bitcode variants when composed into one. In other words, if the replaced code removes other code blocks, all possible combinations having it will be in the end the same program. In the example case, replacing \texttt{code\_block\_2} by \texttt{mul nsw \%0, 3}, turns \texttt{code\_block\_1} into dead code, thus, later replacements generate the same program variants. The rightmost part of \autoref{example:crow:original:combination} illustrates how for three different combinations, CROW produces the same variant. We call this phenomenon an overlapping.

\lstdefinestyle{nccode}{
        numbers=none,
        firstnumber=2,
        stepnumber=1,
        numbersep=10pt,
        tabsize=4, 
        showspaces=false,
        breaklines=true, 
        showstringspaces=false,
    moredelim=**[is][\btHL]{`}{`},
    moredelim=**[is][{\btHL[fill=black!10]}]{`}{`},
    moredelim=**[is][{\btHL[fill=celadon!40]}]{!}{!}
}

\lstset{
    language=WAT,
    style=nccode,
    basicstyle=\footnotesize\ttfamily,
    columns=fullflexible,
    breaklines=true
}


\begin{code}
    \centering
    \captionof{lstlisting}{\termidx{Wasm }program variants generated from program \autoref{CExample}.}\label{example:crow:variants:wasm}
    \lstset{numbers=none}
    \noindent\begin{minipage}[t]{.45\linewidth}
    \begin{lstlisting}[xleftmargin=1em,escapechar=?]
func $f (param i32) (result i32)
   local.get 0
    `i32.const 2`
    `i32.mul`
    `local.get 0`
    `i32.add`

        \end{lstlisting}
\begin{lstlisting}[xleftmargin=1em,escapechar=?]
func $f (param i32) (result i32)
    local.get 0
    !local.get 0!
    !i32.add!
    `local.get 0`
    `i32.add`

                \end{lstlisting}
    \end{minipage}\hfill
    \noindent\begin{minipage}[t]{.45\linewidth}
\begin{lstlisting}[xleftmargin=1em,escapechar=?]
func $f (param i32) (result i32)
    local.get 0
    !i32.const 1!
    !i32.shl!
    `local.get 0`
    `i32.add`

    \end{lstlisting}
\begin{lstlisting}[xleftmargin=1em,escapechar=?]
func $f (param i32) (result i32)
    local.get 0
    !i32.const 3!
    !i32.mul!

            \end{lstlisting}
        \end{minipage}
\end{code}




One might think that a reasonable heuristic could be implemented to avoid such overlapping cases. Instead, we have found it easier and faster to generate the variants with the combination of the replacement and check their uniqueness after the program variant is compiled. This prevents us from having an expensive checking for overlapping inside the CROW code. Still, this phenomenon calls for later optimizations in future works.

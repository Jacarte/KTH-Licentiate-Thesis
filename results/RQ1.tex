
\section{\rqone}
\label{results:rq1}


\newcommand{\rossetapopulationunique}{2678}
\newcommand{\rossetapopulation}{809900}
\newcommand{\diversifiedsodium}{85}
\newcommand{\diversifiedqrcode}{32}
\newcommand{\libsodiumpopulation}{4272}
\newcommand{\libsodiumpopulationunique}{3805}
\newcommand{\qrpopulation}{6369}
\newcommand{\qrpopulationunique}{3314}


\newcommand{\allmewediversified}{\diversifiedsodium + \diversifiedqrcode}
\newcommand{\allmewepopulation}{\libpopulation + \qrpopulation}

As we describe in \autoref{rq1:method}, our first research question aims to answer how to artifically generate \wasm program variants. 
This section is organized as follows. First we present the general results calculating the \emph{\corpuspopulationsizename}(\autoref{metric:rq1:corpus_pop}) and \emph{\corpusuniquepopulationsizename}(\autoref{metric:rq1:corpus_pop_unique}) for each corpus. 
Second, we discuss the challenges and limitations in program variants generation. Finally, we illustrate the most common code transformations performed by our approach and answer RQ1.

\subsection*{Program's populations}

We summarize the results in \autoref{table:crow:general_results}.
The table illustrates the corpus name, the number of functions to diversify, the number of successfully diversified functions (functions with at least one artificially created variant), the cumulative number of variants (\emph{\corpuspopulationsizename}) and the cumulative number of unique variants (\emph{\corpusuniquepopulationsizename}).

% General results on the number of diversified functions per corpus
We produce at least one unique program variant for $239/303{}$ single function programs for \corpusrosetta with one hour for a diversification timeout. For the rest of the programs ($64/303{}$), the timeout is reached before CROW can find any valid variant. 
In the case of \corpussodium and \corpusqrcode, we produce variants for $\py{\diversifiedsodium}/\py{\libsodiumfunctions}$ and $\py{\diversifiedqrcode}/\py{\qrcodefunctions}$ functions respectively, with 5 minutes per function as timeout. The rest of the functions resulted in timeout before finding function variants or produce no variants. For all programs in all corpora, we achieve $356/3021$ successfully diversified functions, representing a $11.78\%$ of the total.
% General growing factor
As the four and fifth columns show, the number of artifically created variants and the number of unique variants are larger than the original number of functions by one order of magnitude. In the case of \corpusrosetta, the corpus population size is close to one million of programs.


\todo{Elaborate on uniqueness...}

\todo{LOW: add histogram on variant sizes}

{
    \renewcommand{\arraystretch}{1.6}
\begin{table}[h]
    %\small
    \centering
    %\setlength\minrowclearance{1.0pt}
        \begin{tabular}[t]{ l | r | l  | l | r}
     Corpus & \#Functions & \# Diversified &  \# Variants & \#  Unique  Variants \\
        \hline   

        \corpusrosetta & \fromjson{data/crow_corpus.json}{.[0].functions} & 239 & \rossetapopulation  &  \rossetapopulationunique \\
        \hline


        \corpussodium & \libsodiumfunctions & \diversifiedsodium & \libsodiumpopulation & \libsodiumpopulationunique     \\
        \hline

        \corpusqrcode & \qrcodefunctions & \diversifiedqrcode  &  \qrpopulation & \qrpopulationunique \\

        \hline\hline

         & \pypy{\libsodiumfunctions + \qrcodefunctions + 303 } & \pypy{239 + \diversifiedqrcode +\diversifiedsodium}  & \pypy{\qrpopulation + \rossetapopulation + \libsodiumpopulation} & \pypy{\qrpopulationunique + \rossetapopulationunique + \libsodiumpopulationunique}   \\
        \end{tabular}
    
        \caption{General program's populations statistics. The table is composed by the name of the corpus, the number of functions, the number of succesfully diversified functions, the cumulative number of generated variants and the cumulative number of unique variants.}
        \label{table:crow:general_results}
\end{table}
}


\subsection*{Challenges for automatic diversification}
\label{rq1:challenges}

% Low diversified functions
We have observed a remarkable difference between the number of successfully diversified functions versus the number of failed-to-diversify functions (third column of \autoref{table:crow:general_results}). Our approach successfully diversified $239/303$, $85/869$ and $32/1849$ of the original functions for \corpusrosetta, \corpussodium and \corpusqrcode respectively.  The main reason of this phenomenon is the set timeout for CROW. 
%\todo{so what? what does this mean for practitioners?: Setting up the timeout affects the capacity of our approach to generate variants. For our corpora, a low timeout implies a low number of diversified functions.}

% High populatio size
We have noticed a remarkable difference between the number of diversified functions for each corpus, \rossetapopulation\ programs for \corpusrosetta 4272 for \corpussodium and 6369 for \corpusqrcode. The corpus population size for \corpusrosetta is two orders of magnitude larger compared to the other two corpora. The reason behind the large number of variants for \corpusrosetta is that, after certain time, our approach starts to combine the code replacements to generate new variants. However, looking at the fifth column, the number of unique variants have the same order of magnitude for all corpora.
The variants generated out of the combination of several code replacements are not necessarily unique. Some code replacements can dominate over others, generating the same \wasm programs.

% Why large unique count in low time ? 
A low timeout offers more unique variants compared to the population size despite the low number of diversified functions, like the \corpussodium and \corpusqrcode cases. This happens because, CROW first generates variants out of single code replacements and then starts to combine them. Thus, more unique variants are generated in the very first moments of the diversification process with CROW.

%Moreover, we look at the cases that yield a few variants per function. There is no direct correlation between the number of identified codes for replacement and the number of unique variants. Therefore, 

Apart from the timeout and the combination of variants phenomena, we manually analyze programs, searching for properties attempting to the generation of program variants using CROW.
We identify another challenge for diversification.
We have observed that our approach searches for a constant replacement for more than $45\%$ of the instructions of each function while constant values cannot be inferred. For instance, \todo{should be first briefly defined for the reader: constant values cannot be inferred. can you elaborate on the importance of constant values for diversification.} for memory load operations because our tool is oblivious to a memory model. 

%\emph{2) Combination computation}  The overlap between code replacements, is a second factor that limits the number of unique variants. We can generate a high number of variants, but not all replacement combinations are necessarily unique.

%\todo{Add all the found examples here}


\subsection*{Properties for large diversification}

We manually analyzed the programs to study the critical properties of programs producing a high number of variants.
This reveals one key factor that favors many unique variants: the presence of bounded loops. In these cases, we synthesize variants for the loops by replacing them with a constant, if the constant inferring is successful. Every time a loop constant is inferred, the loop body is replaced by a single instruction. This creates a new, statically different program. The number of variants grows exponentially if the function contains nested loops for which we can successfully infer constants. 
 

A second key factor for synthesizing many variants relates to the presence of arithmetic. The synthesis engine used by our approach, effectively replaces arithmetic instructions with equivalent instructions that lead to the same result. For example, we generate unique variants by replacing multiplications with additions or shift left instructions (\autoref{add:example}). Also, logical comparisons are replaced, inverting the operation and the operands (\autoref{cmp:examples}). Besides, our implementation can use overflow and underflow of integers to produce variants (\autoref{overflow:example}), using the intrinsics of the underlying computation model.

{
\begin{code}
    \footnotesize
    \lstdefinestyle{nccode}{
        numbers=none,
        firstnumber=2,
        stepnumber=1,
        numbersep=10pt,
        tabsize=4, 
        showspaces=false,
        breaklines=true, 
        showstringspaces=false,
        moredelim=**[is][\btHL]{`}{`},
        moredelim=**[is][{\btHL[fill=black!10]}]{`}{`},
        moredelim=**[is][{\btHL[fill=celadon!40]}]{!}{!}
    }

    \lstset{
        language=WAT,
        style=nccode,
        basicstyle=\footnotesize\ttfamily,
        columns=fullflexible,
        breaklines=true
    }
    \noindent\begin{minipage}[b]{0.32\linewidth}
        \captionof{lstlisting}{Diversification through arithmetic expression replacement.}\label{add:example}
        \noindent\begin{minipage}[t]{0.46\linewidth}
            \begin{lstlisting}
local.get 0
`i32.const 2`
`i32.mul`
            \end{lstlisting}
        \end{minipage}%
        \hfill\noindent\begin{minipage}[t]{0.46\linewidth}
            
            \begin{lstlisting}
local.get 0
!i32.const 1!
!i32.shl!
            \end{lstlisting}
        \end{minipage}
    \end{minipage}\hfill%
    \begin{minipage}[b]{0.31\linewidth}
        \captionof{lstlisting}{Diversification through inversion of comparison operations.}\label{cmp:examples}
        \begin{minipage}[t]{.46\linewidth}
            \begin{lstlisting}
`local.get 0`
`i32.const 10`
`i32.gt_s`
            \end{lstlisting}
        \end{minipage}\hfill\begin{minipage}[t]{.46\linewidth}
           
            \begin{lstlisting}
!i32.const 11!
!local.get 0!
!i32.le_s!
            \end{lstlisting}
        \end{minipage}%
        
        
    \end{minipage}\hfill\noindent
    \noindent\begin{minipage}[b]{0.32\linewidth}
        \captionof{lstlisting}{Diversification through overflow of integer operands.}\label{overflow:example}
        \noindent\begin{minipage}[t]{0.46\linewidth}
            \begin{lstlisting}
`i32.const 2`
i32.mul
\end{lstlisting}
        \end{minipage}%
        \hfill\noindent\begin{minipage}[t]{0.46\linewidth}
            
            \begin{lstlisting}
i32.const 2
i32.mul
!i32.const!
!  -2147483647!
!i32.mul!
            \end{lstlisting}
        \end{minipage}
    \end{minipage}
    \end{code}
}

\todo{Add an example and refer to Cohen when machine code is generated, show that jumps are changed}

\begin{tcolorbox}[title=Answer to RQ1.,boxrule=2pt,arc=.3em,boxsep=1.5mm]
    We can provide diversification for 11.78\% of the programs in our corpora. Constant inferring, complemented with the high presence of arithmetic operations and bounded loops in the original program increased the number of program variants. 
    %\todo{Remove or elaborate: Nevertheless, the combination of code replacements in our approach is not a determinant factor to provide a large number unique program variants.}
\end{tcolorbox}



\begin{comment}

While our work is very limited by the provided corpora, it can be easily extended to other\dots

An application that benefits from the ablity of CROW is the large amount of generated variants. 


- CHeckLong questions for 80%
- Stress how to do experiments in Software Engineering, the theory behind how to do this in CS.
- The objects of the experiments are programs that ... 
- Motivate the corpora selection, size, can be ported, security sensitive, etc,
- Do not mention CROW in the selection criteria.
- The motivation is not related to tools, only to concepts
- Move paragraph after listing of coprpora to before.
- 
%The proposed methodology can generate program variants that are syntactically different from their original versions. We have shown that CROW generates diversity among the binary code variants using semantically equivalent code transformations. We identified the properties that original programs should have to provide a handful number of variants. Besides, we enumerated the challenges faced to provide automatic diversification by retargeting a superoptimizer.

%In the next chapter, we evaluate the assessment of the generated variants answering to what extent the artificial programs are different from the original in terms of static difference, execution behavior, and preservation.

\end{comment}
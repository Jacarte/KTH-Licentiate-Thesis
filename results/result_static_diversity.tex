{
    \renewcommand{\arraystretch}{1.6}
\begin{table}[h]
    \small
    \centering
    %\setlength\minrowclearance{1.0pt}
        \begin{tabular}[t]{ l | c | r  | r  c }
        Corpus & \#Functions & \# Diversified & \# Variants & Population growing factor  \\
        \hline   

        \corpusrosetta & \fromjson{data/crow_corpus.json}{.[0].functions} & \textbf{239} (\py{100*239/303}\%) & 1906  & \py{1906/303}   \\
        \hline


        \corpussodium & \libsodiumfunctions & \textbf{\diversifiedsodium} (\py{100*\diversifiedsodium/\libsodiumfunctions}\%) & \libsodiumpopulation  & \py{\libsodiumpopulation/\libsodiumfunctions}    \\
        \hline

        \corpusqrcode & \qrcodefunctions & \textbf{\diversifiedqrcode} (\py{100*\diversifiedqrcode/\qrcodefunctions}\%) & \qrpopulation  & \py{\qrpopulation/\qrcodefunctions}   \\

        \hline\hline

         & \py{\libsodiumfunctions + \qrcodefunctions + 303 } & \py{239 + \diversifiedqrcode +\diversifiedsodium} (\py{100*{239 + \diversifiedqrcode +\diversifiedsodium}/{\libsodiumfunctions + \qrcodefunctions + 303}}\%) & \py{\qrpopulation + 1906 + \libsodiumpopulation}  & \py{{\qrpopulation + 1906 + \libsodiumpopulation}/{\libsodiumfunctions + \qrcodefunctions + 303}}  \\
        \end{tabular}
    
        \caption{General program's populations statistics. The table is composed by the name of the corpus, the number of functions, the number of succesfully diversified functions, the number of non-diversified functions and the cumulative number of variants.}
        \label{table:crow:general_results}
\end{table}
}
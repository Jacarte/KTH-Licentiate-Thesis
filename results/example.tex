
\subsection{Example}
\label{section:crow:example}
 Let us illustrate how CROW works with the simple example code in \autoref{CExample}. The \texttt{f} function calculates the value of $2 * x + x$ where \texttt{x} is the input for the function.  CROW compiles this source code and generates the intermediate LLVM bitcode in the left most part of \autoref{example:crow:original:llvm}. CROW potentially finds two code blocks to look for variants, as the right-most part of \autoref{example:crow:original:llvm} shows.

% snippet of code showing the detection of code blocks
    
\begin{code}
    \lstset{language=C,
    basicstyle=\small\ttfamily,caption={C function that calculates the quantity $2x + x$},label=CExample}
    \begin{lstlisting}[style=CStyle]
int f(int x) { 
    return 2 * x + x; 
}    
    \end{lstlisting}
    
\end{code}

\lstdefinelanguage{LLVM}
    {morekeywords={i32,mul,align,nsw,add,load,store,define,br, ret, shl, ret},
    sensitive=false,
    morecomment=[l]{;},
    morecomment=[s]{;}{;},
    morestring=[b],
}
\lstdefinestyle{nccode}{
    numbers=left,
    tabsize=4,
    showspaces=false,
    breaklines=true, 
    showstringspaces=false,
    moredelim=**[is][{\btHL[fill=black!10]}]{`}{`},
    moredelim=**[is][{\btHL[fill=celadon!40]}]{!}{!}
}
\lstset{
    language=LLVM,
    style=nccode,
    %basicstyle=\small\ttfamily,
    columns=fullflexible,
    breaklines=true
}


\begin{code}
    \centering
    \captionof{lstlisting}{LLVM's intermediate representation program, its extracted instructions and replacement candidates. Gray highlighted lines represent original code, green for code replacements. }\label{example:crow:original:llvm}
    \lstset{numbers=none}
    \noindent\begin{minipage}[t]{.33\linewidth}
    \centering
    \begin{lstlisting}[xleftmargin=1em,escapechar=?]
    define i32 @f(i32) {

    ?\tikzmarkWS{2}{code 2}{11.5}{10}{3.5cm}?
    ?\tikzmarkWS{1}{code 1}{11.5}{3.5}{3.0cm}?
    %2 = mul nsw i32 %0,2
    %3 = add nsw i32 %0,%2 

    ret i32 %3
    }
    
    define i32 @main() {
    %1 = tail call i32 @f(i32 10)
    ret i32 %1
    }
    \end{lstlisting}
    \end{minipage}%\hfill%
    \begin{minipage}[t]{.32\linewidth}
        \begin{lstlisting}[xleftmargin=1em,escapechar=?]
?Replacement candidates for code\_1?

`%2 = mul nsw i32 %0,2`

!%2 = add nsw i32 %0,%0!

!%2 = shl nsw i32 %0, 1:i32!
    \end{lstlisting}
    \end{minipage}%\hfill%
    \begin{minipage}[t]{.32\linewidth}
        \lstdefinestyle{nccode}{
        tabsize=4, 
        showspaces=false,
        breaklines=true, 
        showstringspaces=false,
        moredelim=**[is][{\btHL[fill=black!10]}]{`}{`},
        moredelim=**[is][{\btHL[fill=celadon!40]}]{!}{!}
        }
        \lstset{
            language=LLVM,
            style=nccode,
            columns=fullflexible,
            breaklines=true,
            belowcaptionskip=1pt,
            abovecaptionskip=1pt,
        } 
        \begin{lstlisting}[name={B},escapechar=?]
?Replacement candidates for code\_2?

`%3 = add nsw i32 %0,%2`

!%3 = mul nsw %0, 3:i32!
        \end{lstlisting}
    \end{minipage}
    
\end{code}





\begin{code}
    \centering
    \captionof{lstlisting}{Candidate code replacements combination. Orange highlighted code illustrate replacement candidate overlapping.}\label{example:crow:original:combination}
    \lstset{numbers=none}
    \noindent\begin{minipage}[t]{.5\linewidth}
    \begin{lstlisting}[xleftmargin=1em,escapechar=?]
`%2 = mul nsw i32 %0,2`
`%3 = add nsw i32 %0,%2`

!%2 = add nsw i32 %0,%0!
`%3 = add nsw i32 %0,%2`

!%2 = shl nsw i32 %0, 1:i32!
`%3 = add nsw i32 %0,%2`

    \end{lstlisting}
    \end{minipage}%\hfill%
    \begin{minipage}[t]{.5\linewidth}
        \lstdefinestyle{nccode}{
        tabsize=4, 
        showspaces=false,
        breaklines=true, 
        showstringspaces=false,
        moredelim=**[is][{\btHL[fill=black!10]}]{`}{`},
        moredelim=**[is][{\btHL[fill=celadon!40]}]{!}{!},
        moredelim=**[is][{\btHL[fill=weborange!40]}]{'}{'}
        }
        \lstset{
            language=LLVM,
            style=nccode,
            columns=fullflexible,
            breaklines=true,
            belowcaptionskip=1pt,
            abovecaptionskip=1pt,
        } 
        \begin{lstlisting}[xleftmargin=1em,escapechar=?]
'%2 = mul nsw i32 %0,2'
!%3 = mul nsw %0, 3:i32!

'%2 = add nsw i32 %0,%0'
!%3 = mul nsw %0, 3:i32!

'%2 = shl nsw i32 %0, 1:i32'
!%3 = mul nsw %0, 3:i32!

    \end{lstlisting}
    \end{minipage}
\end{code}


\begin{tikzpicture}[remember picture,overlay]
%\path (2.north) edge[<-, bend left] (1.north);
%\path[draw, ->] (3.west) edge[<-, bend left] (2.west);
%\path (4.west) edge[<-, bend left] (3.west);
%\path (1.south) edge[<-, bend left] (4.south);

%\path (2.east) edge[<-, bend left, blue] (5.north);
%\path (3.east) edge[<-, bend right, olive] (2.east);
%\path (1.east) edge[<-, bend left, black] (replall1.west);
%\path (2.east) edge[<-, bend left, black] (replall2.west);
%\path (rep11.east) edge[<-, bend left, black] (6.east);
%\path (9.east) edge[<-, bend right, black] (4.east);
%\path (7.east) edge[<-, bend right, black] (8.east);
%\path (5.south) edge[<-, bend right, blue] (4.east);
%\path (9.north) edge[<-] (8.south);
%\path (5.south) edge[<-, bend left] (9.south);


%\path (10.north) edge[<-, bend left] (11.north);
%\path (11.south) edge[<-, bend left] (10.south);
%\path (7) edge[<-, bend right] (6.east);
%\path (8) edge[<-, bend right] (7.east);
\end{tikzpicture}


    

CROW, in the exploration stage detects 2 code blocks, \texttt{code\_block\_1} and \texttt{code\_block\_2} as the enclosing boxes in the left most part of \autoref{example:crow:original:llvm} show. CROW synthesizes $2 + 1$ candidate code replacements for each code block respectively as the green highlighted lines show in the right most parts of \autoref{example:crow:original:llvm}.
The baseline strategy of CROW is to generate variants out of all possible combinations of the candidate code replacements, \ie uses the power set of all candidate code replacements.

In the example, the power set is the cartesian product of the found candidate code replacements for each code block, including the original ones, as \autoref{example:crow:original:combination} shows. The power set size results in $6$ potential function variants. Yet, the generation stage would eventually generate $4$ variants from the original program. CROW generated 4 statically different Wasm files, as \autoref{example:crow:variants:wasm} illustrates. This gap between the potential and the actual number of variants is a consequence of the redundancy among the bitcode variants when composed into one. In other words, if the replaced code removes other code blocks, all possible combinations having it will be in the end the same program. In the example case, replacing \texttt{code\_block\_2} by \texttt{mul nsw \%0, 3}, turns \texttt{code\_block\_1} into dead code, thus, later replacements generate the same program variants. The rightmost part of \autoref{example:crow:original:combination} illustrates how for three different combinations, CROW produces the same variant. We call this phenomenon an overlapping.

One might think that a reasonable heuristic could be implemented to avoid such overlapping cases. Instead, we have found it easier and faster to generate the variants with the combination of the replacement and check their uniqueness after the program variant is compiled. This prevents us from having an expensive checking for overlapping inside the CROW code. Still, this phenomenon calls for later optimizations in future works.

\lstdefinestyle{nccode}{
        numbers=none,
        firstnumber=2,
        stepnumber=1,
        numbersep=10pt,
        tabsize=4, 
        showspaces=false,
        breaklines=true, 
        showstringspaces=false,
    moredelim=**[is][\btHL]{`}{`},
    moredelim=**[is][{\btHL[fill=black!10]}]{`}{`},
    moredelim=**[is][{\btHL[fill=celadon!40]}]{!}{!}
}

\lstset{
    language=WAT,
    style=nccode,
    basicstyle=\footnotesize\ttfamily,
    columns=fullflexible,
    breaklines=true
}


\begin{code}
    \centering
    \captionof{lstlisting}{\termidx{Wasm }program variants generated from program \autoref{CExample}.}\label{example:crow:variants:wasm}
    \lstset{numbers=none}
    \noindent\begin{minipage}[t]{.45\linewidth}
    \begin{lstlisting}[xleftmargin=1em,escapechar=?]
func $f (param i32) (result i32)
   local.get 0
    `i32.const 2`
    `i32.mul`
    `local.get 0`
    `i32.add`

        \end{lstlisting}
\begin{lstlisting}[xleftmargin=1em,escapechar=?]
func $f (param i32) (result i32)
    local.get 0
    !local.get 0!
    !i32.add!
    `local.get 0`
    `i32.add`

                \end{lstlisting}
    \end{minipage}\hfill
    \noindent\begin{minipage}[t]{.45\linewidth}
\begin{lstlisting}[xleftmargin=1em,escapechar=?]
func $f (param i32) (result i32)
    local.get 0
    !i32.const 1!
    !i32.shl!
    `local.get 0`
    `i32.add`

    \end{lstlisting}
\begin{lstlisting}[xleftmargin=1em,escapechar=?]
func $f (param i32) (result i32)
    local.get 0
    !i32.const 3!
    !i32.mul!

            \end{lstlisting}
        \end{minipage}
\end{code}





In this chapter, we investigate to what extent the artifically created variants are different. We propose a methodology to compare the program variants both statically and during runtime. Besides, we present a novel study on code preservation, demonstrating that the code transformations introduced by CROW are resilient to later compiling transformations during machine code generation. We evaluate the variant's preservation in both existing scenarios for \wasm, browsers and standalone engines.

\section{Metrics}

In this section we propose the metrics used along this chapter to answer RQ2. We define the metrics to compare an original program and its variants statically and during runtime. Besides, we proposed the metrics to compare program variants preservation.

\subsection{Static}

To measure the static difference between programs, we compare their bytecode instructions using a global alignment approach. In a previous work of us  we empirically demonstrated that programs semantic can be detected out of its natural diversity \citationneeded. We compare the \wasm of each program and its variant using Dynamic Time Warping (DTW) \cite{Maia08usinga}. DTW computes the global alignment between two sequences. It returns a value capturing the cost of this alignment, which is actually a distance metric. The larger the DTW distance, the more different the two sequences are.

\todo{Add and example here ?}

\begin{metric}{dt\_static:}\label{metric:static1}
	Given two programs $P_X$ and $V_X$ written in $X$ code, dt\_static($P_X$, $V_X$), computes the DTW distance between the corresponding program instructions for representation $X$. \\
	
	A dt\_static($P_X$, $V_X$) of $0$ means that the code of both the original program and the variant  is the same, i.e., they are statically identical in the representation $X$. The higher the value of dt\_static, the more different the programs are in representation X. \\

	Notice that for comparing \wasm programs and its variants, the metric is the instantiation of \DTWStatic with $X=WebAssembly$.
\end{metric}

\subsection{Program traces and execution times}

We measure the difference between programs at runtime by evaluating their execution trace, at function and instruction level. Also, we include the measuring of the execution time of the programs. Besides, we compare their execution times.

\todo{Replace and explain the stack trace as stack operations}

\begin{metric}{\DTW{}:}\label{metric:stack}
	Given a program P, a \tool generated variant P' and $T$ a trace space ($T \in \{Function, Instruction \}$) \DTW{}(P,P',T), computes the DTW distance between the traces collected during their execution in the $T$ space. A \DTW{} of $0$ means that both traces are identical. \\ 
	
	The higher the value, the more different the traces. 
\end{metric}


\begin{metric}{Execution time:}\label{metric:time}
	Given a \wasm program P, the execution time is the time spent to execute the binary.
\end{metric}

\subsection{Variants preservation}

The last metric is needed because \wasm is an intermediate language and compilers use it to produce machine code. For program variants, this means that compilers can undo artificial introduced transformations, for example, through optimization passes. When a code transformation is maintained from the first time it is introduced to the final machine code generation is  a preserved variant. 

Part of the contributions of this thesis are our strategies to prevent reversion of code transformations. We take engineering decision regarding this in all the stages of the CROW workflow. We disable all optimizations inside CROW in the generation of the \wasm binaries. This prevents the LLVM toolchain used to remove some introduced transformations. However, the LLVM toolchain applies optimizations by default, such as constant folding or logical operations' normalization. As we illustrate previously, these are some transformations found and applied by CROW. We modified the LLVM backend for \wasm to avoid this reversion during the creation of Wasm binaries.
This phenomenon is sometimes bypassed by diversification studies when they are conducted at high-level. As another contribution, we conduct a study on preservation for both scenarios where Wasm is used, browsers and standalone engines. In

The final metric corresponds to the preservation study. We compare two programs to be different under the \wasm representation and under the machine code representation after they are compiled through a collection of selected \wasm engines. We use two instances of \autoref{metric:static1} for two different code representations, \wasm and x86. The key property we consider is as follows: \\


We only take into account the x86 representation after the \wasm code is compiled to the machine code. 
This decision is not arbitrary, according to the study of \todo{Paper on binary diff survey}, any conclusion carried out by comparing two program binaries under a specific target can be extrapolated to another target for the same binaries.

%\section{Setup}

\section{Evaluation}

To answer RQ2 we use the same corpora proposed and evaluated in \autoref{chapter:generation}, \textbf{CROW prime} and \textbf{MEWE prime}. We analyze the variants generated in the RQ1 answering. \todo{Add the numbers here}

% Static
\subsection{Static comparison}
For each function on the corpora, we compare the sequence of instructions of each variant with the initial program and the other variants. We obtain the \autoref{metric:static} values for each program-variant \wasm pair code. We compute the DTW distances with STRAC~\cite{Cabrera19}. 

% Dynamic
\subsection{Dynamic comparison}
To compare program and variants behavior during runtime, we analyze all the unique program variants generated by \tool in a pairwise comparison. 
We use SWAM\footnote{\url{https://github.com/satabin/swam}} to collect the function and instruction traces. SWAM is a \wasm interpreter that provides functionalities to capture the dynamic information of \wasm program executions including the stack operations. We compute the DTW distances with STRAC~\cite{Cabrera19}. 

Furthermore, we collect the execution time, \autoref{metric:time}, for all programs and their variants. We execute each program or variant \todo{XXX} times and we compare the collected execution times using a Mann-Withney test \citationneeded.

\subsection{Preservation}

We collect \autoref{metric:preservation} for all programs and their generated variants. We use the engines listed in \autoref{assesment:preservation:engines}.

\todo{ We can add the other binaries }

\begin{table}[h]
	\begin{tabular}{p{2cm} | p{9cm} }
	%\hline
	Name & Properties \\
	\hline
	V8 \citationneeded & V8  is the engine used by Chrome and NodeJS to execute JavaScript and \wasm. \todo{Explain compilation process} \\
	\hline
	wasmtime \citationneeded & Wasmtime is a standalone runtime for WebAssembly. This engine is used by the Fastly platform to provide Edge-Cloud computing services. \todo{Explain compilation process}  \\		
	\end{tabular}
	\caption{Wasm engines used during the diversification assessment study. The table is composed by the name of the engine and the description of the compilation process for them.}
	\label{assesment:preservation:engines}
\end{table}

%\subsection{Setup}


\section{Results}

\section{Results}
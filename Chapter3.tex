
\chapter{Methodology} 
\label{chapter:generation}

% Define some numbers here for the autmation of the tables
\newcommand{\libsodiumfunctions}{687}
\newcommand{\qrcodefunctions}{1840}
\newcommand{\allmewefunctions}{\libsodiumfunctions + \qrcodefunctions}

% Execute a python script for small calculations
\newcommand{\py}[1]{\input{|python3 interpreter.py #1}}

%This chapter investigates whether we can artificially create program variants through semantically equivalent code transformations. We propose a framework to generate program variants functionally equivalent to their original.
%We introduce the retargeting of a superoptimizer, using its exhaustive search strategy to provide semantically equivalent code transformations. 
%The presented methodology and transformation tool, CROW, are contributions to this thesis.
%We evaluate the usage of CROW on two corpora of open-source and nature diverse programs. 



\section{Evaluation}
\label{section:crow:exp_setup}

We use CROW, the tool described at \autoref{section:crow}, to answer RQ1. This section describes the corpora of original programs that we pass to CROW for the sake of variants generation. Besides, we describe our metrics and finalize the section by discussing the results.

\subsection{Corpora}
\label{section:crow:corpora}

We answer RQ1 with two corpora of programs appropriate for our experiments. The first corpus, \textbf{CROW prime}, is part of the CROW contribution \cite{}. The second corpus, \textbf{MEWE prime}, is part of the MEWE contribution \cite{}. In \autoref{table:corpora} we summarize the selection criteria, and we mention each corpus properties. With both corpora we evaluate CROW with a total of $303 + \py{\allmewefunctions}$ functions. 

\subsection{Metric}
 
To assess our approach's ability to generate \wasm binaries that are statically different, we compute the number of unique variants generated by CROW for each original function. 
We compare the \wasm program and its variant using the \texttt{md5} hash of each function byte stream as a metric for uniqueness.


\begin{table}[h]
    \renewcommand{\arraystretch}{1.5}
    \footnotesize
    \centering
    \begin{tabular}{p{1cm} p{6cm} p{5cm}}
        Corpus name & Selection criteria & Corpus Description \\
        \midrule
        \textbf{CROW prime} & We take programs from the  Rosetta Code project\footnote{\url{http://www.rosettacode.org/wiki/Rosetta_Code}}. 
        %This website hosts a curated set of solutions for specific programming tasks in various  programming languages.
        %It contains a wide range of tasks, from simple ones, such as adding two numbers, to complex algorithms like a compiler lexer. 
        We first collect all C programs from Rosetta Code, which represents $989$ programs as of 01/26/2020. 
        
        We then apply a number of filters: the programs should successfully compile, they should not require user inputs, the programs should terminate and should not provide in non-deterministic results.  
        The result of the filtering is a corpus of 303 C programs  &  All programs have a single function in terms of source code. These programs range from $7$ to $150$ lines of code and solve a variety of problems, from the \textit{Babbage} problem to  \textit{Convex Hull} calculation. \\
        \hline
        \textbf{MEWE prime} & We select two mature and typical edge-cloud computing projects for this corpus.
        The projects are selected based on their suitability for  diversity synthesis with CROW, \ie the projects should have the ability to collect their modules in LLVM intermediate representation
        
        %, suitability for deployment on the Fastly infrastructure (the project should be easily portable Wasm/WASI and compatible with the Rust Fastly API). 

        The selected projects are: \textbf{libsodium}, an encryption, decryption, signature and password hashing library which can be ported to WebAssembly and \textbf{qrcode-rust}, a QrCode and MicroQrCode generator written in Rust. 
        
        &  The evaluated projects contain in total \py{\allmewefunctions} functions, \libsodiumfunctions\ for libdosium and \qrcodefunctions\ for qrcode-rust. The functions range between 10 ad 127700 lines of code. \\
    \end{tabular}
    \caption{Corpora description. The table is composed by the name of the corpus, the selection criteria and the stats the programs in each corpus.}
    \label{table:corpora}
\end{table}

\todo{Move text out of the table}



\subsection{Setup}

CROW's workflow synthesizes program variants with an enumerative strategy. All possible programs that can be generated for a given language (LLVM in the case) are constructed and verified for equivalence.
There are two parameters to control the size of the search space and hence the time required to traverse it.
On the one hand, one can limit the size of the variants. On the other hand, one can limit the set of instructions used for the synthesis. On the other hand, in our experiments, we use between $1$ instruction (only additions) and $60$ instructions (all supported instructions in the synthesizer).


These two configuration parameters allow the user to find a trade-off between the number of variants that are synthesized and the time taken to produce them. In \autoref{table:corpora:setup} we listed the configuration for both corpora. For the current evaluation, given the size of the corpus, we set the exploration time to 1 hour maximum per function for \textbf{CROW PRIME}. In the case of \textbf{MEWE prime}, we set the timeout to 5 minutes per function in the exploration stage. We set all 60 supported instructions in CROW for both \textbf{CROW prime} and \textbf{MEWE primer} corpora.

\begin{table}[H]
    \renewcommand{\arraystretch}{1.2}
    \centering
    \begin{tabular}{l | l l}
        \midrule
        CORPUS & Exploration timeout & Max. instructions \\
        \hline
        CROW prime & 1h & 60 \\
        MEWE prime & 5m & 60 \\
    \end{tabular}
    \caption{CROW tweaking for variants generation. The table is composed by the name of the corpus, the timeout parameter and the count of allowed instructions during the synthesis process.}
    \label{table:corpora:setup}
\end{table}

\let\cleardoublepage\clearpage
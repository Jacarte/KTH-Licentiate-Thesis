\chapter{Methodology} 
\label{chapter:method}

\pagestyle{plain}
% Define some numbers here for the autmation of the tables
\newcommand{\libsodiumfunctions}{869}
\newcommand{\qrcodefunctions}{1849}
\newcommand{\allmewefunctions}{\libsodiumfunctions + \qrcodefunctions}

% Execute a python script for small calculations
\newcommand{\py}[1]{\input{|python3 interpreter.py #1}}
\newcommand{\fromjson}[2]{\input{| jq -r '#2' #1}}

\newcommand{\corpusrosetta}{\fromjson{data/crow_corpus.json}{.[0].name}}
\newcommand{\corpussodium}{Libsodium\xspace}
\newcommand{\corpusqrcode}{QrCode\xspace}


\newcommand{\DTWStatic}{dt\_static\xspace}
\newcommand{\DTW}{dt\_dyn\xspace}
\newcommand{\tool}{CROW\xspace}


%This chapter investigates whether we can artificially create program variants through semantically equivalent code transformations. We propose a framework to generate program variants functionally equivalent to their original.
%We introduce the retargeting of a superoptimizer, using its exhaustive search strategy to provide semantically equivalent code transformations. 
%The presented methodology and transformation tool, CROW, are contributions to this thesis.
%We evaluate the usage of CROW on two corpora of open-source and nature diverse programs. 
In this chapter, we present our methodology to answer the research questions enunciated in \autoref{chapter:introduction}.
We investigate three research questions. In the first question, we artificially generate \wasm program variants, and we qualitatively compare the stati differences between variants. 
Our second research question focuses on comparing their behavior during their execution.
The final research question evaluates the feasibility of using the program variants in security-sensitive environments such as Edge-Cloud computing proposing a multivariant execution approach.

The main objective of this thesis is to study the feasibility of automatically creating programs variant out of preexisting program sources or case studies. To achieve this goal,
we use the empirical method \cite{}, proposing a solution and evaluating it through quantitative analyzes in case studies. We follow an iterative and incremental approach that start with the selection of program sources to be amplified through automated diversification. 
We share the same corpora of programs to answer all our research questions. We first enunciate the corpora of programs. Then, for each research question, we establish the metrics, set the configuration for the experiments, and describe the protocol for them.


% Our approach lies under \textit{Design Science} \cite{Runeson2020}, in terms of empirical validation, the scope of the design knowledge gained in a study can be extended by systematically extending the scope of the valudation in subsequent studies. Thus, the size of our corpora can be extended to increase the knowledge of the research area.


\section{Corpora}
\label{section:crow:corpora}

Our experiments assess the impact of artificially created diversity. For such reason, the first step is to build a suitable corpus of programs to generate program variants. We then use the generated variants to study their static, dynamic, and security properties.  We answer all our research questions with three corpora of programs appropriate for our experiments. In \autoref{table:corpora} we listed the corpus name, the number of programs inside the corpus, the total number of functions, the range of lines of code, and the original location of the corpus. In the following, we describe the filtering and description of each corpus.

We build our corpora in an escalating strategy. The first corpus should illustrate the feasibility of CROW to generate program variants out of simple programs in terms of code size. The latter two corpora study the impact of CROW on more extensive real-world programs, including one project meant for security-sensitive applications. Overall, all corpora are considered to come along the LLVM pipeline, having the input for CROW from C/C++ source code or LLVM bitcodes. We base this decision on the previous experimental work of Lehman \etal \citationneeded. This work shows that more than 70\% of all \wasm programs come out of LLVM based tooling.

\begin{enumerate}
    \item \textbf{\corpusrosetta} corpus is part of the CROW contribution \cite{}. We take programs from the  Rosetta Code project\footnote{\url{http://www.rosettacode.org/wiki/Rosetta_Code}}. This website hosts a curated set of solutions for specific programming tasks in various  programming languages. It contains a wide range of tasks, from simple ones, such as adding two numbers, to complex algorithms like a compiler lexer.  We first collect all C programs from Rosetta Code, which represents $989$ programs as of 01/26/2020.  We then apply a number of filters: the programs should successfully compile, they should not require user inputs, the programs should terminate and should not provide in non-deterministic results. The result of the filtering is a corpus of 303 C programs. All programs have a single function in terms of source code. These programs range from $7$ to $150$ lines of code and solve a variety of problems, from the \textit{Babbage} problem to  \textit{Convex Hull} calculation.
    \item \textbf{\corpussodium} is part of both CROW and MEWE contributions \cite{} \cite{}. This project is an encryption, decryption, signature and password hashing library which can be ported to WebAssembly.
    We selected 5 programs or endpoints to answer our research questions. These endpoints have between $8$ and $2703$ lines of code per function.  The project is selected based on their suitability for  diversity synthesis with CROW, \ie the project should have the ability to collect its modules in LLVM intermediate representation and the project should be easily portable Wasm/WASI.

    \item \textbf{\corpusqrcode} is part of the MEWE contribution. This project is a QrCode and MicroQrCode generator written in Rust. We selected 2 programs or endpoints to answer our research questions. These endpoints have between $4$ and $725$ lines of code per function. As \corpussodium, we select this project due to its suitability for diversity synthesis with CROW.
\end{enumerate}


%The first corpus, \textbf{CROW prime}, . The second corpus, \textbf{MEWE prime}, is part of the MEWE contribution \cite{}. In \autoref{table:corpora} we summarize the selection criteria, and we mention each corpus properties. With both corpora we evaluate CROW with a total of $303 + \py{\allmewefunctions}$ functions. 

\begin{table}[h]
    \renewcommand{\arraystretch}{1.0}
    \small
    \centering
    \begin{tabular}{l  | l | l | l | p{2.8cm}}
        Corpus name & No. programs & No. functions & LOC range & Location \\
        \midrule
            % CROW
            \corpusrosetta &
            \fromjson{data/crow_corpus.json}{.[0].programs} &
            \fromjson{data/crow_corpus.json}{.[0].functions}  & 
            \fromjson{data/crow_corpus.json}{.[0].min_lines} - 
            \fromjson{data/crow_corpus.json}{.[0].max_lines} & 
            \fromjson{data/crow_corpus.json}{.[0].url} \\
        \hline
        \corpussodium & 
        5 & 
        \fromjson{data/allinone.multivariant.bc.massive.sodium.json}{.total_functions}  &
        \fromjson{data/allinone.multivariant.bc.massive.sodium.json}{.min_llvm_loc} - \fromjson{data/allinone.multivariant.bc.massive.sodium.json}{.max_llvm_loc}  &   
        \url{https://github.com/jedisct1/libsodium }\\
        \hline
        \corpusqrcode & 
        2 & 
        \fromjson{data/allinone.multivariant.bc.massive.qr.json}{.total_functions}  & 
        \fromjson{data/allinone.multivariant.bc.massive.qr.json}{.min_llvm_loc} - \fromjson{data/allinone.multivariant.bc.massive.qr.json}{.max_llvm_loc}   & 
        \url{https://github.com/kennytm/qrcode-rust} \\
        % Total stats
        \hline
        \hline
        \textbf{Total} & 
        \py{
        5 + 2 + 303} &   
        \py{ 303 + \qrcodefunctions + \libsodiumfunctions} &  
        &     \\

    \end{tabular}
    \caption{Corpora description. The table is composed by the name of the corpus, the number of programs, the number of functions, the lines of code range and the location of the corpus.}
    \label{table:corpora}
\end{table}


\section{\rqone}
\label{rq1:method}
This research question investigates whether we can artificially generate program variants for \wasm. Concretely, we use CROW to generate program variants from an original program, written in C/C++ or directly passing an LLVM bitcode module to it. Therefore, the first step and research question is related to the ability of CROW to generate a handful number of program variants. Our intuition is that the larger the number of generated variants, the better the program variants' security and reliability properties can offer.



\begin{figure*}[h]
    \centering
    \includegraphics[height=2in]{diagrams/Rq1.pdf}
    \caption{Simplification of the program variants generation workflow.}
    \label{diagrams:protocol:rq1}
\end{figure*}


In \autoref{diagrams:protocol:rq1} we simplify the workflow to generate \wasm program variants. We pass each function of the corpora to CROW. To answer RQ1, we study the outcome of this pipeline, the generated variants. 

\subsection{Metrics}

To assess our approach's ability to generate \wasm binaries that are statically different, we compute the number of unique variants generated by CROW for each original function of each corpus. 



\begin{metric}{Population size $S(P)$:}\label{metric:md5sum}
    Given a program P and its generated variants $V$, the population size metric is defined as.\\
    $$
        S(P)=|V|
    $$

    Notice that, the variant population includes P as an instance.
\end{metric}

A program and its variants compose what we call a program's population. Notice that all proposed metrics over programs and their variants make sense only at the population level. Therefore, it only makes sense to compare semantically equivalent programs, i.e., from the same population. Along with this work, we use the term "program's population" to refer to a program and its variants.

\subsection{Protocol}



One design property of CROW is that all possible programs that can be generated for a given language (LLVM in this case) are constructed and verified for equivalence. Thus, there are two parameters to control the size of the search space and hence the time required to traverse it.
On the one hand, one can limit the size of the variants. On the other hand, one can limit the set of instructions used for the synthesis. In our experiments for RQ1, we use all the $60$ supported instructions in the synthesizer.

The former parameter allows us to find a trade-off between the number of variants that are synthesized and the time taken to produce them. For the current evaluation, given the size of the corpus, we set the exploration time to 1 hour maximum per function for \corpusrosetta. In the cases of \corpussodium\ and\ \corpusqrcode, we set the timeout to 5 minutes per function in the exploration stage. The decision behind the usage of lower timeout for \corpussodium
and \corpussodium is motivated by the properties listed in \autoref{table:corpora}. The latter two corpora are remarkably larger in terms of the number of instructions and functions count. 

We pass each of the $303 + \libsodiumfunctions + \qrcodefunctions$ functions in the corpora to CROW, as \autoref{diagrams:protocol:rq1} illustrates, to synthesize program variants. We then calculate \autoref{metric:md5sum} for each program's population and conclude by answering RQ1.



\section{\rqtwo}
\label{rq2:method}


\begin{figure*}[h]
    \centering
    \includegraphics[width=\linewidth]{diagrams/Rq2.pdf}
    \caption{Dynamic analysis for RQ2.}
    \label{diagrams:protocol:rq2}
\end{figure*}

In this second research question, we investigate to what extent the artificially created variants are dynamically different between them and in comparison to the original program. To conduct this research question, we could separate our experiments into two fields as \autoref{diagrams:protocol:rq2} illustrates: static analysis and dynamic analysis. 
The static analysis focuses on the appreciated differences among the program variants, as well as between the variants and the original program. We perform the static analysis in answering RQ1 in \autoref{rq1:method}. 
With RQ2, we focus on the last category, the dynamic analysis of the generated variants. This decision is supported because dynamic analysis complements RQ1 and, it is essential to provide a full understanding of diversification.
We use the original functions from the \corpusrosetta corpus described in \autoref{section:crow:corpora} and their variants generated to answer RQ1. 
We use only \corpusrosetta to answer RQ2 because this corpus is composed of simple programs that can be executed directly without user interaction, \ie we only need to call the interpreter passing the \wasm\ binary to it. 
To dynamically compare programs and their variants, we execute each program on each programs' population to collect and execution times. We define execution trace and execution time in the following section.

\subsection*{Metrics}
\label{rq2:metrics}

We compare the execution traces of two any programs of the same population with a global alignment metric. We propose a global alignment approach using Dynamic Time Warping (DTW).
Dynamic Time Warping \cite{NEEDLEMAN1970443} computes the global alignment between two sequences. It returns a value capturing the cost of this alignment, which is a distance metric. The larger the DTW distance, the more different the two sequences are.
DTW has been used for comparing traces in different domains. For software, De A. Maia \etal\ \cite{Maia08usinga} proposed to identify similarity between programs from execution traces.
As we discussed in \autoref{sota:wasm}, a theoretical \wasm\ engine perform \texttt{push} and \texttt{pop} operations when the program instructions are executed. Therefore, in our experiments, we define the execution traces as the sequence of the stack operations during the execution of the \wasm\ program. 
In the following text, we define the $\DTW$ metric. 
 
%\todo{before, define this and give an illutrative listing plus, says how you collect those traces, that's part of the protocol: between the stack operation traces }

\begin{metric}{\DTW{}:}
\label{metric:stack}
\normalfont 
	Given two programs P and P' from the same program's population, \DTW{}(P,P'), computes the DTW distance collected during their execution. \\
	A \DTW{} of $0$ means that both traces are identical.
	The higher the value, the more different the traces. 
\end{metric}



Moreover, we use the execution-time distribution of the programs in the population to complement the answer to RQ2. For each program pair in the programs' population, we compare their execution-time distributions. We define the execution time as follows:

\begin{metric}{Execution time:}\label{metric:time}
    \normalfont 
	Given a \wasm\ program P, the execution time is the time spent to execute the binary.
\end{metric}



%\subsection{Variants preservation}

\subsection*{Protocol}

% Dynamic
To compare program and variants behavior during runtime, we analyze all the unique program variants generated to answer RQ1 in a pairwise comparison using the value of aligning their execution traces (\autoref{metric:stack}). We use SWAM\footnote{\url{https://github.com/satabin/swam}} to execute each program and variant to collect the stack operation traces. SWAM is a \wasm\ interpreter that provides functionalities to capture the dynamic information of \wasm\ program executions, including the virtual stack operations.
% \todo{Can the reader runderstand that? We want to remark that we only collect the stack operation traces due to the memory-agnosticism of our approach to generate variants. Our approach does not change the memory-like operations of the original code.}

Furthermore, we collect the execution time, \autoref{metric:time}, for all programs and their variants. We compare the collected execution-time distributions between programs using a Mann-Withney U test \cite{mann1947} in a pairwise strategy.

%\todo{Maybe the first time that Mann-Withney is mentioned I should describe what it is}

 


\section{\rqthree}
\label{rq3:method}

\newcommand{\mewe}{MEWE\xspace}

In the last research question, we study whether the created variants can be used in real-world applications and what properties offer the composition of the variants as multivariant execution binaries. For this purpose, we build multivariant binaries to be deployed at the Edge. The process of \emph{mixing} multiple variants into one multivariant binary is an essential contribution of the thesis that is presented in details in \citationneeded. RQ3 focuses on analyzing the impact of this contribution on execution times. To answer RQ3, we use the variants generated for the programs of the \corpussodium and \corpusqrcode corpora, $2 + 5$ programs involving $ \libsodiumfunctions + \qrcodefunctions$ functions respectively. We illustrate the protocol to answer RQ3 in \autoref{diagrams:protocol:rq3}.


\begin{figure*}[h]
    \centering
    \includegraphics[width=\linewidth]{diagrams/Rq3.pdf}
    \caption{Multivariant binary creation and workflow for RQ3.}
    \label{diagrams:protocol:rq3}
\end{figure*}

We assess the ability of multivariant binaries to exhibit random execution paths when executed on edge. We check the diversity of execution traces gathered from the execution of a multivariant binary. The traces are collected from all edge nodes to assess Multivariant Execution (MVE) worldwide. Finally, we measure the differences for the execution times on the edge. Then, we discuss how multivariant binaries contribute to less predictable timing side-channels.

\subsection*{Metrics}

We use the execution time of the multivariant binaries to answer RQ3. We compare the distribution of the execution times for multivariant binary and the original program. 

\subsection*{Protocol}


We run the experiments to answer RQ3 on the Edge. We deploy and execute the original and the multivariant binaries on 64 edge nodes located around the world.
These edge nodes usually have an arbitrary and heterogeneous composition in architecture and CPU models.

We collect 100k execution times for each binary, both the original and multivariant binaries.
We perform a Mann-Withney U test \cite{mann1947} to compare both execution time distributions. 
If the P-value is lower than 0.05, two compared distributions are different.

\todo{Improve this...}





\section{Conclusions}

This chapter presents the methodology we follow to answer our three research questions. We first describe and propose the corpora of programs used in this work. We propose to measure the ability of CROW to generate variants out of \py{303  + \libsodiumfunctions + \qrcodefunctions} functions of our corpora. Then, we suggest using the generated variants to study to what extent they offer different observable behavior through  dynamic analysis. Finally, we propose a protocol to study the impact of the composition variants in a multivariant binary deployed at the Edge. Nevertheless, we enumerate and enunciate the properties and metrics that might lead us to answer the impact of automatic diversification for \wasm programs. In the next chapter, we present and discuss the results obtained with this methodology.
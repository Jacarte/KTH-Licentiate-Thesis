    
\begin{code}
    \lstset{language=C,
    basicstyle=\small\ttfamily,caption={C function that calculates the quantity $2x + x$},label=CExample}
    \begin{lstlisting}[style=CStyle]
int f(int x) { 
    return 2 * x + x; 
}    
    \end{lstlisting}
    
\end{code}

\lstdefinelanguage{LLVM}
    {morekeywords={i32,mul,align,nsw,add,load,store,define,br, ret, shl, ret},
    sensitive=false,
    morecomment=[l]{;},
    morecomment=[s]{;}{;},
    morestring=[b],
}
\lstdefinestyle{nccode}{
    numbers=left,
    tabsize=4,
    showspaces=false,
    breaklines=true, 
    showstringspaces=false,
    moredelim=**[is][{\btHL[fill=black!10]}]{`}{`},
    moredelim=**[is][{\btHL[fill=celadon!40]}]{!}{!}
}
\lstset{
    language=LLVM,
    style=nccode,
    %basicstyle=\small\ttfamily,
    columns=fullflexible,
    breaklines=true
}


\begin{code}
    \centering
    \captionof{lstlisting}{LLVM's intermediate representation program, its extracted instructions and replacement candidates. Gray highlighted lines represent original code, green for code replacements. }\label{example:crow:original:llvm}
    \lstset{numbers=none}
    \noindent\begin{minipage}[t]{.33\linewidth}
    \centering
    \begin{lstlisting}[xleftmargin=1em,escapechar=?]
    define i32 @f(i32) {

    ?\tikzmarkWS{2}{code 2}{11.5}{10}{3.5cm}?
    ?\tikzmarkWS{1}{code 1}{11.5}{3.5}{3.0cm}?
    %2 = mul nsw i32 %0,2
    %3 = add nsw i32 %0,%2 

    ret i32 %3
    }
    
    define i32 @main() {
    %1 = tail call i32 @f(i32 10)
    ret i32 %1
    }
    \end{lstlisting}
    \end{minipage}%\hfill%
    \begin{minipage}[t]{.32\linewidth}
        \begin{lstlisting}[xleftmargin=1em,escapechar=?]
?Replacement candidates for code\_1?

`%2 = mul nsw i32 %0,2`

!%2 = add nsw i32 %0,%0!

!%2 = shl nsw i32 %0, 1:i32!
    \end{lstlisting}
    \end{minipage}%\hfill%
    \begin{minipage}[t]{.32\linewidth}
        \lstdefinestyle{nccode}{
        tabsize=4, 
        showspaces=false,
        breaklines=true, 
        showstringspaces=false,
        moredelim=**[is][{\btHL[fill=black!10]}]{`}{`},
        moredelim=**[is][{\btHL[fill=celadon!40]}]{!}{!}
        }
        \lstset{
            language=LLVM,
            style=nccode,
            columns=fullflexible,
            breaklines=true,
            belowcaptionskip=1pt,
            abovecaptionskip=1pt,
        } 
        \begin{lstlisting}[name={B},escapechar=?]
?Replacement candidates for code\_2?

`%3 = add nsw i32 %0,%2`

!%3 = mul nsw %0, 3:i32!
        \end{lstlisting}
    \end{minipage}
    
\end{code}





\begin{code}
    \centering
    \captionof{lstlisting}{Candidate code replacements combination. Orange highlighted code illustrate replacement candidate overlapping.}\label{example:crow:original:combination}
    \lstset{numbers=none}
    \noindent\begin{minipage}[t]{.5\linewidth}
    \begin{lstlisting}[xleftmargin=1em,escapechar=?]
`%2 = mul nsw i32 %0,2`
`%3 = add nsw i32 %0,%2`

!%2 = add nsw i32 %0,%0!
`%3 = add nsw i32 %0,%2`

!%2 = shl nsw i32 %0, 1:i32!
`%3 = add nsw i32 %0,%2`

    \end{lstlisting}
    \end{minipage}%\hfill%
    \begin{minipage}[t]{.5\linewidth}
        \lstdefinestyle{nccode}{
        tabsize=4, 
        showspaces=false,
        breaklines=true, 
        showstringspaces=false,
        moredelim=**[is][{\btHL[fill=black!10]}]{`}{`},
        moredelim=**[is][{\btHL[fill=celadon!40]}]{!}{!},
        moredelim=**[is][{\btHL[fill=weborange!40]}]{'}{'}
        }
        \lstset{
            language=LLVM,
            style=nccode,
            columns=fullflexible,
            breaklines=true,
            belowcaptionskip=1pt,
            abovecaptionskip=1pt,
        } 
        \begin{lstlisting}[xleftmargin=1em,escapechar=?]
'%2 = mul nsw i32 %0,2'
!%3 = mul nsw %0, 3:i32!

'%2 = add nsw i32 %0,%0'
!%3 = mul nsw %0, 3:i32!

'%2 = shl nsw i32 %0, 1:i32'
!%3 = mul nsw %0, 3:i32!

    \end{lstlisting}
    \end{minipage}
\end{code}


\begin{tikzpicture}[remember picture,overlay]
%\path (2.north) edge[<-, bend left] (1.north);
%\path[draw, ->] (3.west) edge[<-, bend left] (2.west);
%\path (4.west) edge[<-, bend left] (3.west);
%\path (1.south) edge[<-, bend left] (4.south);

%\path (2.east) edge[<-, bend left, blue] (5.north);
%\path (3.east) edge[<-, bend right, olive] (2.east);
%\path (1.east) edge[<-, bend left, black] (replall1.west);
%\path (2.east) edge[<-, bend left, black] (replall2.west);
%\path (rep11.east) edge[<-, bend left, black] (6.east);
%\path (9.east) edge[<-, bend right, black] (4.east);
%\path (7.east) edge[<-, bend right, black] (8.east);
%\path (5.south) edge[<-, bend right, blue] (4.east);
%\path (9.north) edge[<-] (8.south);
%\path (5.south) edge[<-, bend left] (9.south);


%\path (10.north) edge[<-, bend left] (11.north);
%\path (11.south) edge[<-, bend left] (10.south);
%\path (7) edge[<-, bend right] (6.east);
%\path (8) edge[<-, bend right] (7.east);
\end{tikzpicture}


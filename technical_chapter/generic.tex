\section{Global approach}
\label{tech:generic}


The work of Hilbig et al. \cite{Hilbig2021AnES} in 2021 influences our design decisions. According to their work, 70\% of the \wasm\ binaries in the wild are created with LLVM-based compilers. Therefore, we provide artificial software diversity for \wasm\ through LLVM. 
Other solutions would have been to diversify at the source-code level or the \wasm\ binary level. However, these facts would limit the applicability of our work.
Our approach is more general as diversification also will work for other LLVM backends.

LLVM is a compound of three main components \cite{llvmofficialweb}. First, the frontend (compilers such as clang and rustc) converts the program source code to LLVM intermediate representation (LLVM IR). Second, optimization and transformation processes improve the LLVM IR. Third and final, the backend component is in charge of generating the target machine code. In \autoref{diagrams:generic} we show how we use the LLVM pipeline in our contributions, which are highlighted as dashed squares.

\begin{figure*}[h]
    \includegraphics[width=\linewidth]{diagrams/architecture.pdf}
    \caption{Generic workflow to create \wasm\ program variants.}
    \label{diagrams:generic}
\end{figure*}



The global workflow in \autoref{diagrams:generic} starts by receiving the source code. Then the LLVM frontend transforms it into LLVM IR representation \step{1}. 
We alter the LLVM pipeline that compiles source code to Wasm by introducing a diversifier component.  

The diversifier generates LLVM IR variants from the output of the frontend \step{2}. 
The LLVM IR variants are inputs for our customized Wasm backend. 
The diversifier and the custom Wasm LLVM backend compose CROW, which creates \wasm\ program variants out of a source code program \step{3}. 
In addition, an orthogonal tool comes from the generation of LLVM IR variants at Step~\step{2}. MEWE  \cite{MEWE}, merges and creates multivariant binaries to provide MVE for \wasm\ \step{4}.  
%In \autoref{section:mewe} we describe MEWE in details.

%Our first technical contribution, CROW  \cite{CROW}, includes the implementation of the diversifier for LLVM and the customized \wasm\ backend. 


%Besides, as we discussed previously, our intention is also to study the impact of our contributions in edge computing and distributed systems and the top edge computing execution platforms, e.g. Cloudflare and Fastly, mostly take \wasm\ binaries as input. 
%LLVM, on the contrary, supports different languages, with a rich ecosystem of frontends it it can reliably be retargeted to \wasm, thanks to the corresponding mature component in the LLVM toolchain. In addition, the LLVM ecosystem as a whole is very active, providing us with many different tools to facilitate our research endeavour.
%In this chapter we summarize the technical details for our two contributions. \autoref{section:crow} we dissect the main components CROW \cite{CROW} implementation. Finally, in \autoref{section:mewe}, we describe the technical details of our second contribution, MEWE \cite{MEWE}.
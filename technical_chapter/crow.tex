
\section{CROW}
\label{section:crow}

This section describes CROW, our first contribution. CROW is a tool tailored to create semantically equivalent \wasm variants out of a single program, either C/C++ and Rust code or LLVM bitcode.
In \autoref{diagrams:crow}, we describe the workflow of CROW to create program variants.


\subsection*{Overview}

\begin{figure*}[h]
    \includegraphics[width=\linewidth]{diagrams/generation/crow.drawio.pdf}
    \caption{CROW workflow to generate program variants. CROW takes C/C++ source codes or LLVM bitcodes to look for code blocks that can be replaced by semantically equivalent code and generates program variants by combining them.}
    \label{diagrams:crow}
\end{figure*}

Figure \ref{diagrams:crow} highlights the main two stages of the CROW's workflow, \textit{exploration} and \textit{combining}. The workflow starts by compiling the input program into the LLVM bitcode using clang from the source code. During the \emph{exploration} stage, CROW takes an LLVM bitcode and, for its code blocks, produces a collection of code replacements that are functionally equivalent to the original program. In the following, we enunciate the definitions we use along with this work for a code block, functional equivalence, and code replacement. 


\begin{definition}{Block (based on Aho \etal \cite{ahodragonbook}):}\label{def:code-block}
    Let $P$ be a program. A block $B$ is a grouping of declarations and statements in $P$ inside a function $F$. 
\end{definition}


\begin{definition}{Functional equivalence modulo program state (based on Cohen \etal \cite{cohen1993operating}):}
    \label{def:functional-equivalence}
    Let $B_1$ and $B_2$ be two code blocks according to \autoref{def:code-block}. We consider the program state before the execution of the block, $S_i$, as the input and the program state after the execution of the block, $S_o$, as the output. $B_1$ and $B_2$ are functionally equivalent if given the same input $S_i$ both codes produce the same output $S_o$.
\end{definition}

\begin{definition}{Code replacement:}
    \label{def:code-replacement}
    Let $P$ be a program and $T$ a pair of code blocks $(B_1, B_2)$. $T$ is a candidate code replacement if $B_1$ and $B_2$ are both functionally equivalent as defined in \autoref{def:functional-equivalence}.
    Applying $T$ to $P$ means replacing $B_1$ by $B_2$. The application of $T$ to $P$ produces a program variant $P'$ which consequently is functionally equivalent to $P$.     
\end{definition}

We have found the work of Jacob \etal \cite{jacob2008superdiversifier} on superdiversification the best approach to generate artificial diversity at a fine-grained level. 
We use their seminal work to implement CROW based on two main reasons. First, the code replacements generated by this technique outperform diversification strategies based on hand-written transformation rules and it is fully automatic. Second, the existence of a battle tested superoptimizer for LLVM, Souper \cite{Sasnauskas2017Souper:Superoptimizer}. 
We implement the \emph{exploration} stage by retargeting Souper. The main objective of Souper is to find the best (smallest) possible program, we modify it to keep the intermediate solutions in their searching algorithm to generate program variants.  
We prevent the superoptimizer from synthesizing instructions that have no correspondence in the \wasm backend for the sake of reducing the searching space for variants. Besides, we disable the majority of the pruning strategies of Souper for the sake of more variants.
In addition, we also modify the LLVM compiler, by disabling all optimizations in the \wasm backend that could reverse the superoptimizer transformations, such as constant folding and instructions normalization.


CROW operates at the code block level, taking them from the functions defined inside the input LLVM bitcode module. In addition, the retargeted superoptimizer is in charge of finding the potential places in the original code blocks where a replacement can be applied. Finally, we use the enumerative synthesis strategy of the retargeted superoptimizer to generate code replacements.
The code replacements generated through synthesis are verified, according to \autoref{def:functional-equivalence}, by internally using a theorem prover. 

%\todo{We disable cost restrictions and the LLVM backend optimizations...maybe for the assesment RQ ?}

In the \emph{combining} stage, CROW combines the candidate code replacements to generate different LLVM bitcode variants, selecting and merging the code replacements. 
Then, a variant bitcode is compiled into a \wasm binary if requested. Finally, CROW generates the variants from all possible combinations of code replacements as the power set of all code replacements.  

\subsection*{Constant inferring as a new diversification transformation}

\todo{Constant inferring and why is important and novel in our work.}


\subsection*{Example}
\label{section:crow:example}
 Let us illustrate how CROW works with the simple example code in \autoref{CExample}. The \texttt{f} function calculates the value of $2 * x + x$ where \texttt{x} is the input for the function.  CROW compiles this source code and generates the intermediate LLVM bitcode in the left most part of \autoref{example:crow:original:llvm}. CROW potentially finds two code blocks to look for variants, as the right-most part of \autoref{example:crow:original:llvm} shows.

% snippet of code showing the detection of code blocks

 \begin{code}
    \lstset{language=C,caption={C function that calculates the quantity $2x + x$},label=CExample}
    \begin{lstlisting}[style=CStyle]
    int f(int x) { 
        return 2 * x + x; 
    }
    
    int main(void) { return f(10); }
    \end{lstlisting}
    
    \end{code}

 \lstdefinelanguage{LLVM}
    {morekeywords={i32,mul,align,nsw,add,load,store,define,br, ret, shl, ret},
    sensitive=false,
    morecomment=[l]{;},
    morecomment=[s]{;}{;},
    morestring=[b]",
}
\lstdefinestyle{nccode}{
    numbers=left,
    tabsize=4,
    showspaces=false,
    breaklines=true, 
    showstringspaces=false,
    moredelim=**[is][{\btHL[fill=black!10]}]{`}{`},
    moredelim=**[is][{\btHL[fill=celadon!40]}]{!}{!}
}
\lstset{
    language=LLVM,
    style=nccode,
    %basicstyle=\small\ttfamily,
    columns=fullflexible,
    breaklines=true
}


\begin{code}
    \centering
    \captionof{lstlisting}{LLVM's intermediate representation program and its code blocks.}\label{example:crow:original:llvm}
    \lstset{numbers=none}
    \noindent\begin{minipage}[b]{.55\linewidth}
    \centering
    \begin{lstlisting}[xleftmargin=1em,escapechar=?]
    define i32 @f(i32) {

     ?\tikzmarkWS{2}{code block 2}{12.5}{10}{4.5cm}?
     ?\tikzmarkWS{1}{code block 1}{13.5}{4}{3.5cm}?
     %2 = mul nsw i32 %0,2
     %3 = add nsw i32 %0,%2 

     ret i32 %3
    }
    
    define i32 @main() {
     %1 = tail call i32 @f(i32 10)
     ret i32 %1
    }
    \end{lstlisting}
    \end{minipage}\hfill%
    \noindent\begin{minipage}[b]{.4\linewidth}
    \vspace{-8cm}
        \lstdefinestyle{nccode}{
          tabsize=4, 
          showspaces=false,
          breaklines=true, 
          showstringspaces=false,
        moredelim=**[is][{\btHL[fill=black!10]}]{`}{`},
        moredelim=**[is][{\btHL[fill=celadon!40]}]{!}{!}
        }
        \lstset{
            language=LLVM,
            style=nccode,
            columns=fullflexible,
            breaklines=true,
            belowcaptionskip=1pt,
            abovecaptionskip=1pt,
        }
        \vfill%
        \begin{lstlisting}[label={ref:block1} ,name={A},escapechar=?]
    ?\tikzmarkPROBE{5}{bb4}{-4}{10}? 
    %2 = mul nsw i32 %0,2
        \end{lstlisting}
        \begin{lstlisting}[name={B},escapechar=?]
    ?\tikzmarkPROBE{6}{bb4}{-4}{8}? 
    %2 = mul nsw i32 %0,2
    %3 = add nsw i32 %0,%2
        \end{lstlisting}
    \end{minipage}
\end{code}


\begin{tikzpicture}[remember picture,overlay]
%\path (2.north) edge[<-, bend left] (1.north);
%\path[draw, ->] (3.west) edge[<-, bend left] (2.west);
%\path (4.west) edge[<-, bend left] (3.west);
%\path (1.south) edge[<-, bend left] (4.south);

%\path (2.east) edge[<-, bend left, blue] (5.north);
%\path (3.east) edge[<-, bend right, olive] (2.east);
\path (1.east) edge[<-, bend right, black] (5.east);
\path (2.east) edge[<-, bend right, black] (6.east);
%\path (6.east) edge[<-, bend right, black] (3.east);
%\path (9.east) edge[<-, bend right, black] (4.east);
%\path (7.east) edge[<-, bend right, black] (8.east);
%\path (5.south) edge[<-, bend right, blue] (4.east);
%\path (9.north) edge[<-] (8.south);
%\path (5.south) edge[<-, bend left] (9.south);


%\path (10.north) edge[<-, bend left] (11.north);
%\path (11.south) edge[<-, bend left] (10.south);
%\path (7) edge[<-, bend right] (6.east);
%\path (8) edge[<-, bend right] (7.east);
\end{tikzpicture}

    

CROW, in the exploration stage detects 2 code blocks, \texttt{code\_block\_1} and \texttt{code\_block\_2} as the enclosing boxes in the left most part of \autoref{example:crow:original:llvm} show. CROW synthesizes $2 + 1$ candidate code replacements for each code block respectively as the green highlighted lines show in the right most parts of \autoref{example:crow:original:llvm}.
The baseline strategy of CROW is to generate variants out of all possible combinations of the candidate code replacements, \ie uses the power set of all candidate code replacements.

In the example, the power set is the cartesian product of the found candidate code replacements for each code block, including the original ones, as \autoref{example:crow:original:combination} shows. The power set size results in $6$ potential function variants. Yet, the generation stage would eventually generate $4$ variants from the original program. CROW generated 4 statically different Wasm files, as \autoref{example:crow:variants:wasm} illustrates. This gap between the potential and the actual number of variants is a consequence of the redundancy among the bitcode variants when composed into one. In other words, if the replaced code removes other code blocks, all possible combinations having it will be in the end the same program. In the example case, replacing \texttt{code\_block\_2} by \texttt{mul nsw \%0, 3}, turns \texttt{code\_block\_1} into dead code, thus, later replacements generate the same program variants. The rightmost part of \autoref{example:crow:original:combination} illustrates how for three different combinations, CROW produces the same variant. We call this phenomenon a candidate code replacement overlapping.

One might think that a reasonable heuristic could be implemented to avoid such overlapping cases. Instead, we have found it easier and faster to generate the variants with the combination of the replacement and check their uniqueness after the program variant is compiled. This prevents us from having an expensive checking for overlapping inside the CROW code. Still, this phenomenon calls for later optimizations in future works.

\input{snippets/wasm_codes.tex}

\chapter{Background \& State of the art}

This chapter discusses state of the art in the areas of \emph{\wasm} and \emph{Software Diversification}. In \autoref{sota:wasm} we discuss the \wasm language, its motivation, how \wasm binaries are generated, language specification and security related issues. In \autoref{sota:sota}, we present a summary of Software Diversification and its foundational concepts. In addition, we compare and locate our contributions againts state of the art related works.  
We select the discussed works by their novelty, critical insights and representativeness of their used techniques. 
We finalize the chapter by stating our novel contributions. 


%In \autoref{sota:wasm} we describe the context in which this dissertation is based, \wasm. We include both usage scenarios for Wasm and its main security issues. In \autoref{sota:diversification} we discuss the main diversification techniques in the wild, we introduce the concept of superoptimization used in this work, and we end the section by highlighting superdiversification as the cornerstone of our contributions are based. In \autoref{sota:randomization} we mention close works to runtime diversification and how diversification can be used to construct resilient binaries. 

\section{\wasm}
\label{sota:wasm}

\todo{Intro}

\subsection*{Overview}

% Javascript was first, and then more attempts.
Over the past decades, JavaScript has been used in the majority of the browsers vendors to allow client-side scripting. However, due to the complexity of the language and to gain in performance, several approaches appeared to introduce different languages.  For example, Java applets were introduced on web pages at \todo{XXX}, Adobe added ActionScript later on \todo{XXX}, Microsoft made an attempt with ActiveX \todo{XXX}. All these attempts failed, most of them due to security issues and lack of consensus on the community. 

% asm.js and the demonstration of bad language patterns
In 2014, Emscripten introduced a strict subset of JavaScript, asm.js, to allow low level code such as C to be compiled to JavaScript \todo{GG}. Asm.js was first implemented as an LLVM backend. This approach came with the benefits of having all the ahead-of-time optimizations from LLVM, gaining in twice the performance on browser clients \citationneeded compared to standard JavaScript code. The main reason why asm.js was faster, is that it limits the language features to those that can be optimized in the LLVM pipeline or those that can be directly translated from the source code. Besides, it removes the majority of the dynamic characteristics of the language, limiting it to numerical types, top-level functions, and one large array in the memory directly accessed as raw data. For example, C code won't be compiled to asm.js with nested functions. Since asm.js is a subset of JavaScript it was compatible with all engines. Asm.js demonstrated that client-code could be improved with the right language design and standarization.

% Limitations of asm.js and the birth of Wasm
The work of Van Es \etal \cite{EsAsm.js} proposed to shrink JavaScript to asm.js in a source-to-source strategy, closing the cycle and extending the fact that asm.js was mainly a compilation target for C/C++ code. Despite their result was encouraging, JavaScript faces several limitations related to the intrinsics of the language. For example, any JavaScript engine requires the parsing and the recompilation of the JavaScript code creating overhead \todo{Cryptic}. We, as many in the community consider asm.js the precursor of \wasm. The \wasm (Wasm) language was first publicly announced in 2015. \wasm is a binary instruction format for a stack-based virtual machine and was officially stated by the work of Haas \etal \cite{Haas_2017} in 2017. The announcement of \wasm marked the first step of standarizing bytecode in the web environment. Wasm is designed to be fast, portable, self-contained and secure, and it outperforms asm.js \citationneeded.

Since then, \wasm adoption has an exponential growth, companies like Adobe recently announced a full online version of Photoshop \citationneeded. Moreover, game companies moved their development from JavaScript to Wasm \citationneeded and stack technologies such as Microsoft implement their virtual machines in Wasm \citationneeded.

%\todo{Benchmarks}

\subsection*{From source to Wasm}

All \wasm programs are compiled ahead of time from source languages. LLVM includes Wasm as a target since version 8, supporting a broad range of frontend languages such as C/C++. The resulting binary, works similarly to a shared library, it includes instruction codes, symbols and exported functions. In \autoref{diagrams:sota:wasm}, we illustrate the workflow of the creation of Wasm binaries and their lifecycle. The process starts by compiling the source code program to Wasm (Step \step{1}). To the best of our knowledge, the best survey Wasm is presented in the Awesome Wasm (\url{https://github.com/mbasso/awesome-wasm}) repo. It is a cumulative GitHub repo which includes references to articles, papers, books, demos, compilers and engines. 

The step \step{2} includes the composition of the glue code to Wasm as JavaScript code. This code will include the harness that the Wasm binary needs. For example, by design, Wasm cannot access the DOM of the page as JavaScript code does and if need it, the functions to interact with the DOM are imported in the Wasm binary during its call. The glue JavaScript code can be manually written, however, compilers like Emscripten, Rust and Binaryen can generate it automatically.

Finally, the third step (Step \step{3}), includes the compilation of the client-side code. Most of the browser engines compile either the Wasm code or JavaScript to machine code. In the case of JavaScript this process involves JIT and hot code replacement during runtime. For Wasm, this process is almost direct, in the case of V8, only two compilation processes are required, a baseline compilation, meant to make the coming binary available as soon as possible and a final one, that makes some lane optimizations. This analysis was validated by conversations with the V8 develop team and by one experimental study in our contributions.  

\begin{figure}[h]
    \centering
    \includegraphics[width=\linewidth]{diagrams/wasm_workflow.pdf}
    \caption{TODO1.}
    \label{diagrams:sota:wasm}
\end{figure}

The Wasm binary might not need external glue code. Thus, depends on the host environment and the binary itself to supply it. This fact encourages the adoption of \wasm further browser engines. For instance, Cloudflare and Fastly adapted their platforms to provide Function as a Service (FaaS) directly with \wasm. In this case the glue code, instead of JavaScript, is provided by any other language stack that the host environment supports.
In 2019, the bytecode alliance team \footnote{\url{https://bytecodealliance.org/}} proposed WebAssembly System Interface (WASI). WASI is the foundation to build Wasm code outside of the browser with a POSIX system interface platform. It standarizes the adoption of \wasm outside web browsers \cite{bryant2020webassembly} in heterogeneous platforms like the Edge or IoT \cite{Narayan2021Swivel,Sledge}. 

% Talk about benchmarks and performance numbers

%Previous studies resulted in performance increasing in terms of bandwidth saving, execution, and process-on-demand spawning \cite{9640153, wen2020wasmachine}. The words of Solomon Hykes \footnote{\url{https://twitter.com/solomonstre/status/1111004913222324225}}, the former CEO of docker, show the impact of WASI: 

%\begin{displayquote}
%\textit{
%    If WASM+WASI existed in 2008, we wouldn't have needed to created Docker. That's how important it is. Webassembly on the server is the future of computing. A standardized system interface was the missing link. Let's hope WASI is up to the task!
%}
%\end{displayquote}

%\todo{How to obtain WebAssembly binaries}

%\todo{Browser workflow}

%\todo{How interpreters work, explain the workflow of V8 as it was validated by the V8 compiler developers}

%\todo{Backend workflow}

%\todo{Add a table with tools and interpreters: V8, SpiderMonkey, wasmtime, wasmer, sledge, warduino }


\subsection*{Instruction Set Architecture}

% General description
\wasm defines its out Instruction Set Architecture (ISA). It is an abstraction close to machine code instructions but agnostic to CPU architectures. Thus, Wasm is platform independent. The ISA of Wasm includes also the necessary components that the binary requires to run in any host engine. These components include the memory model and bytecode instructions.
A Wasm binary has a module as its main component (according to the current roadmap and supported version in all Wasm vendors). A module is composed by 12 \todo{check} types of sections, each of them with an explicit semantic and a specific order inside the module. This makes the compilation to machine code faster. The binary format of Wasm can include custom sections. For example, the work of \todo{Doe} proposed the usage of custom sections to sign binaries for the sake of trusting.

% Example
\begin{code}
    \lstset{language=C,caption={C function that calculates the quantity $2x + x$},label=CExample}
\begin{lstlisting}[style=CStyle]
int f(int x) { 
    return 2 * x + x; 
}

int main(void) { 
    return f(10); 
}
\end{lstlisting}
    
\lstset{
        language=WAT,
        caption={\wasm code  for \autoref{CExample}.},
        style=WATStyle,
        stepnumber=0,
        label=WASMExample}

\begin{lstlisting}
(module
    (type (;0;) (func (param i32) (result i32)))
    (type (;1;) (func (result i32)))
    (func (;0;) (type 0) (param i32) (result i32)
        local.get 0
        local.get 0
        i32.const 2
        i32.mul
        i32.add)
    (func (;1;) (type 1) (result i32)
        i32.const 10
        call 0)
    (export "main" (func 1)))
\end{lstlisting}
\end{code}


% Fast, instructions, fast parsing, compile as it comes

% Portable, unrelated to CPU architectures, some number on size vs JS code

% Self contained loops

% Bound memory, no direct access to the DOM

% Roadmap: threads, SIMD, etc

\todo{Internals of WebAssembly}

\todo{Wasm Semantics: Add a code listing, explain General layout of a binary, Function and signature, Instructions, Control flow, Memory model, Global model, table model}

\todo{Add some sentences from the roadmap}

\todo{Exported function signature allows typing validation before the code is executed}


\subsection*{Wasm security}


\begin{comment}

\todo{Add example...}
\todo{Go slow...}
\todo{Toolchain, explain frontends, general pipeline...}

In this section, we introduce an overview of the motivation for \wasm and its usage. 
\todo{Jumps to fast, avoid !. Sentece 2} Besides, we describe the process to obtain Wasm programs and how this novel technology evolves from being only-browser-based to standalone executions in the backend. Nevertheless, we describe its significant limitations regarding security, that is our main motivation for our research.

\todo{Why Wasm, asm.js, from source to Wasm, mention benchmarks and examples, photoshop, games, etc, data types, memory model, stack modeling, from Wasm to execution, finalize with security, benefits and limitations. Add diagram}
\todo{The need of a virtual machine or a compiler.}
\todo{Why backend, why WASI as an extension not an essential part of Wasm for the browser.}

\todo{Add diagram, for obtaining Wasm and executing them, with references}
% What is Wasm


% How to generate Wasm code
\wasm binaries are obtained from source code like C/C++ or Rust \cite{Hilbig2021AnES}. The \wasm code is further interpreted or compiled ahead of time into machine code by engines such as the browsers. Since version 8, LLVM supports Wasm as a backend opening the door for its vast collection of frontend languages. The LLVM support was encouraged by the seminal work of Zakai \etal with Emscripten.
Emscripten is an open-source tool for compiling C/C++ to the \wasm. It uses LLVM to create Wasm, but it provides support for faster linking to the object files. Instead of all the IR being compiled by LLVM, the object file is prelinked with Wasm in a faster way. 

\subsection*{\wasm for backend execution}



% Why is important to study Wasm
\subsection*{\wasm security and our motivation for diversification}


\wasm is characterized by a robust security model~\cite{WebAssemblySecurity}. It should run inside a sandboxed execution environment that provides protection against common security issues such as data corruption, code injection, and return-oriented programming (ROP). However, WebAssembly is vulnerable under certain conditions, at the execution engine's level \cite{ChromeZero} or the binary itself ~\cite{usenixWasm2020}.
Implementations in both browsers and standalone runtimes~\cite{Narayan2021Swivel} are vulnerable.
This means that if one environment is vulnerable, all the others are vulnerable in the same manner as the same \wasm binary is replicated, turning it into a monoculture problem.


On the other hand, the \wasm environment lacks natural diversity \cite{natural_diversity}. Compared to the work of Harrand \etal \citationneeded, in WebAssembly, one could not use preexisting and different program versions to provide diversification for monoculture solving. In fact, according to the work of Hilbig \etal \cite{Hilbig2021AnES}, the artificial variants created with one of our works contribute to the half of executable and available \wasm binaries in the wild. 


The current limitations on security and the lack of preexisting diversity motivate our work on software diversification as one possible mitigation among the wide range of security countermeasures. 

\end{comment}


\section{Software Diversification}
\label{sota:sota}
%Checkmarck symbol
\def\checkmark{\tikz\fill[scale=0.4](0,.35) -- (.25,0) -- (1,.7) -- (.25,.15) -- cycle;} 

% Commands to refer to the milestones
\newtheoremstyle{sota}% name of the style to be used
  {\topsep}% measure of space to leave above the theorem. E.g.: 3pt
  {\topsep}% measure of space to leave below the theorem. E.g.: 3pt
  {\itshape}% name of font to use in the body of the theorem
  {0pt}% measure of space to indent
  {\bfseries}% name of head font
  {}% punctuation between head and body
  { }% space after theorem head; " " = normal interword space
  {(\thmname{#1}\thmnumber{#2})\textnormal{\thmnote{ (#3)}}}

\def\Gnospace~{G{}}
\theoremstyle{sota}
\newtheorem{goal}{G}
\providecommand*{\definitionautorefname}{\Gnospace}
\newcommand{\goalautorefname}{\Gnospace}


\def\Snospace~{S{}}
\theoremstyle{sota}
\newtheorem{strategy}{S}
\providecommand*{\definitionautorefname}{\Snospace}
\newcommand{\strategyautorefname}{\Snospace}

\def\Unospace~{U{}}
\theoremstyle{sota}
\newtheorem{usage}{U}
\providecommand*{\definitionautorefname}{\Unospace}
\newcommand{\usageautorefname}{\Unospace}

\todo{Esta muy regado ahora.}

\todo{Stress input-output equivalence in the artificial diversification.}

\todo{Mention concrete objective and achievement when citing.}

\todo{emove some strategies, just cite some papers and thats it. Reduce to five transformations. Stick to U2 andd U3. Add techincal stack in the table.}

\todo{Add level of transformation as a dimension. Hoigh level, fine-grained transofmrations, etc}

\todo{Start with COhen, then go form high level to fine granied and from technologies to LLVM and Wasm.}


\todo{Given a collection of pgorams, then enumerate the usages. Second approach, multiple encapsulation in one single program.}

The low presence of defenses implementations for \wasm motivates our work on Software Diversification as a preemptive technique that can help against known and yet unknown vulnerabilities.
Software Diversification has been widely studied in the past decades. This section discusses its state of the art.
% What is Software diversity
Software diversification consists in synthesizing, reusing, distributing, and executing different, functionally equivalent programs. 
According to the survey of Baudry and Monperrus \cite{natural_diversity}, the motivation for software diversification can be separated in five categories: reusability \cite{pohl2005software}, software testing \cite{Chen2010AdaptiveRT}, performance \cite{10.1145/2025113.2025133}, fault tolerance \cite{1659219} and security \cite{cohen1993operating}. Our work contributes to the latter two categories. 

\begin{comment}

\begin{goal}{Fault tolerance and reliability:}
    \label{goal:reliability}
    \normalfont
    Mainly refers to the implementation of independent yet functionally equivalent programs for consensus during execution. Different programs are deployed and executed simultaneously; the final result is selected from all computation results. If some programs fail, the system is still able to respond. For decades, this same idea was applied to hardware. For example, in airplanes is common to have more than one sensor providing the same function. For example, Harrand \etal \cite{harrand2020java} proposed to combine the result of multiple Java decompilers.
\end{goal}

\begin{goal}{Security:}
    \label{goal:security}    
    \normalfont
    Mainly refers to break code reuse attacks \cite{595185} by using diverse functional programs. The main idea is to change the observable behavior of a program by changing its version every time is invoked. Thus, attackers cannot get the same information from a different source. For example, Crane \etal \cite{crane2015thwarting} hardened power side-channels by using diversification of software. On the same topic, Roy \etal \cite{10.1145/3318216.3363338} use preexisting machine learning algorithms to defeat adversarial-like attackers.
\end{goal} 
\end{comment}


%\emph{Testing}(G3) is similar to fault tolerance. The main idea is to use the failures of tests in a collection of similar programs to detect failure patterns \cite{1659219}.
%\emph{Reusability}(G4) refers to reliability in terms of extended functionality.

%\todo{Cite security}, \todo{cite reliability}



%\todo{Define diversification here and motivate its usage. Highlight e use for security.}

%\subsubsection{For security}

\subsection*{Artificial Software Diversity.}

There are two primary sources of software diversification: natural(ND) and artificial diversity(AD) \cite{natural_diversity}. Natural diversity can be controlled or an unpredicted consequence of developing processes. Controlled natural diversity is usually called Design Diversity or N-Version Diversity. It is addressed using engineering decisions \cite{1659219}. In practice, it consists of providing N development teams with the exact requirements. The teams develop N independent versions using different approaches. On the other hand, Natural Diversity can emerge from spontaneous software development processes. To illustrate the Natural Diversity phenomenon, CodeForces\footnote{\url{https://codeforces.com/contest/1667/status/page/2?order=BY_PROGRAM_LENGTH_ASC}} shows more than 350 different and successful solutions in C++ for a single requirements based problem in a single programming contest. 
The software market is an expected source of natural diversity. Sengupta \etal \cite{10.5555/3091125.3091155} used this fact to reach the security goal (\autoref{goal:security}).



% Jump to the need of artificial
Notice that Natural Diversity can rely on itself to escalate, and it is coped by the preexistence of software. This might be a limitation. For example, in the context o this work, the natural diversity for \wasm programs is nearly inexistence \cite{Hilbig2021AnES}. When natural diversity is not enough, it is innate to think that the source for diversification needs to be artificial. Automated software diversification consists of artificially synthesizing software.

% Classification by Cohen
According to the seminal work of Cohen \etal \cite{cohen1993operating} automatic software diversification can be reached by mutation strategies. A mutation strategy is a set of rules to define how a specific piece of software should be changed to provide a different yet functionally equivalent variant. A mutation can be applied at different layers of software lifecycle, from compilation to execution and from source code to executable binary. We have found that the foundation for automatic software diversity has barely changed since Cohen in 1993. Complemented with the work of Baudry and Monperrus \cite{natural_diversity}, we enumerate the strategies for automatic software diversification. 

We use the concept of functional equivalence defined in the seminal work of Cohen \etal \cite{cohen1993operating} as input-output equivalence. Two programs are equivalent if, given identical input, they produce the identical output.

\begin{strategy}{Equivalent arithmetic instructions}
    \label{strategy:S1}
    \normalfont
    Numeric calculations can be expressed theoretically in an infinite number of ways. This strategy is simple but powerful since the complexity of program variants dramatically increases. In terms of overhead, the size of the program variant increases with the size of the replacement. 
    
\end{strategy}


\begin{strategy}{Instruction reordering}
    \label{strategy:S2}
    \normalfont
    This strategy reorders instructions or entire program blocks if they are independent. This strategy generates program variants without affecting their size and execution time. 

\end{strategy}

\begin{strategy}{Variable substitution}
    \label{strategy:S3}
    \normalfont
    This strategy changes the location of variable declarations. It has a lower impact on low-level programs unless compilers resort to symbol tables. It prevents static examination and analysis of parameters and alters memory locations. The strategy should not affect the size of program variants neither their execution time.
\end{strategy}


\begin{strategy}{Adding, changing, removing jumps}
    \label{strategy:S4}
    \normalfont
    This strategy creates program variants by adding, changing, or removing jumps inside the original program. At a high level, this can be reached by loop splitting or by inserting arbitrary branching. This strategy increases the execution time of variants.
\end{strategy}


\begin{strategy}{Adding, changing, removing calls}
    \label{strategy:S5}
    \normalfont
    This strategy is similar to the previous one (\autoref{strategy:S4}). It extends the same idea by adding function calls inside the stack. It is mainly implemented by inlining and de-inlining expressions inside the program. When an expression is inlined, its original instruction is replaced by a function call that performs the same calculation as the original subexpression.
\end{strategy}


\begin{strategy}{Garbage insertion}
    \label{strategy:S6}
    \normalfont
    This strategy adds instructions to the program that are independent of the original sequence of instructions. It extends \autoref{strategy:S1} by supporting more instructions like random memory accesses.
    Dealing with code-reuse attacks, Homescu \etal~\cite{homescu2013profile} propose inserting garbage NOP instructions directly in LLVM IR to generate a variant with a different code layout at each compilation. 
    Jackson \etal \cite{jackson} have explored how to use NOP operations inserted during compiling time to diversify programs. Jackson \etal \cite{jackson}  used the optimization flags of several compilers to generate semantically equivalent binaries out of the same source code. These techniques place the compiler at the core of the diversification technique. %However, this approach is limited by the number of available flags in the compiler implementation and because the optimization is applied in all possible places in the code at the same time.
\end{strategy}

\begin{strategy}{Program layout randomization in memory and stack}
    \label{strategy:S7}
    \normalfont
    This strategy is usually implemented on top of compilers. The generation of the final binary and how it operates its memory is randomized.  %This makes it very hard for an attacker to ignore the key to inject executable code. This breaks the predictability of program execution and mitigates certain exploits. 
\end{strategy}


\begin{strategy}{ISA randomization}
    \label{strategy:S8}
    \normalfont
    This strategy encodes the original program, for example, by using a simple XOR operation over its binary bytestream. This technique is strong against attacks involving the examination of code. It should not affect the size of program variants or their execution times. The encoding/decoding operations are performed only once. 
    Seminal works include instruction-set randomization \cite{Kc03,barrantes2003randomized}
    to create a unique mapping between artificial CPU instructions and real ones.
\end{strategy}


\begin{strategy}{Simulation}
    \label{strategy:S9}
    \normalfont
    It is an interpretation mechanism similar to encoding (\autoref{strategy:S9}), but the execution of programs is delegated to a custom interpreter instead of using preexisting execution hosts. The program is decoded at runtime every time it is invoked. 
    Chew, and Song \cite{Chew02mitigatingbuffer} target operating system randomization. They randomize the interface between the operating system and the user applications:
    the system call numbers, the library entry points (memory addresses), and the stack placement. All those techniques are dynamic, done at runtime using load-time preprocessing and rewriting. 
\end{strategy}


\begin{strategy}{Intermixing}
    \label{strategy:S10}
    \normalfont
    With the existence of more than one program variant, the execution of program variants can be mixed. The decision of which variant executes is decided at runtime. This strategy is the core for randomization, multivariant execution and the execution by consensus defined in \autoref{goal:reliability}. This strategy is 
    complex to implement because the integrity of the memory and stack needs to be stable between programs.
\end{strategy}

\subsection*{Usages of Software Diversity}

We categorize the applications of software diversity in three main usages.

\begin{usage}{N-Version:}
    \label{usage:n-version}
    \normalfont
    More than one independent program variant is executed in one single machine. For example, the work of Sengupta \etal \cite{10.5555/3091125.3091155} proposed to change indistinctly between database engines and backend server implementations in a monolithic web application. Their approach decreases the reach of exploitable CVEs from request to request. 
    In this area, Coppens \etal~\cite{coppens2013feedback} use compiler transformations to diversify software iteratively. Their work aims to prevent reverse engineering of security patches for attackers targeting vulnerable programs.
    Their approach continuously applies a random selection of predefined transformations using a binary diffing tool as feedback. 
    A downside of their method is that attackers are, in theory, able to identify the type of transformations applied and find a way to ignore or reverse them.
    Bathkar et al. \cite{bhatkar03,bhatkar2005efficient} proposed three kinds of randomization transformations: randomizing the base addresses of applications and libraries' memory regions, a random permutation of the order of variables and routines, and the random introduction of random gaps between objects. 
    Dynamic randomization can address different kinds of problems. In particular, it mitigates an extensive range of memory error exploits. 
    Recent work in this field includes stack layout randomization against data-oriented programming \cite{aga2019smokestack}, and memory safety violations \cite{lee2021savior}, as well as a technique to reduce the exposure time of persistent memory objects to increase the frequency of address randomization \cite{xu2020merr}.
\end{usage}


\begin{usage}{Randomization:}
    \label{usage:randomization}
    \normalfont
    One program that decides at runtime which of its variants to execute. Courouss{\'e} \etal~\cite{courousse2016runtime} implement an assembly-like DSL to generate equivalent code at runtime in order to increase protection against side-channel attacks. 
    A randomization technique creates a set of unique executions for the very same program \cite{bhatkar03}. 
    Previous works have attempted to generate diversified variants that are alternated during execution.
    It has been shown to drastically increase the number of execution traces required by a side-channel attack as
    Amarilli \etal~\cite{amarilli2011can} has shown. On the same topic,
    Agosta \etal~\cite{agosta2015meet} and Crane \etal~\cite{crane2015thwarting}
    modify the LLVM toolchain to compile multiple functionally equivalent variants to randomize software control flow at runtime.
\end{usage}

\begin{usage}{Multivariant Execution(MVE):}
    \label{usage:mve}
    \normalfont
    Multiple program variants are composed in one single binary that can execute separately depending on external configuration or in parallel for consensus computation. In 2006, security researchers at the University of Virginia laid the foundations of a novel approach to security that consists in executing multiple variants of the same program, \cite{cox06}. Bruschi et al. \cite{bruschi2007diversified} and Salamat et al. \cite{salamat2007stopping} pioneered the idea of executing the variants in parallel. Subsequent techniques focus on Multivariant Execution for mitigating memory vulnerabilities \cite{lu2018stopping} and other specific security problems incl. return-oriented programming attacks \cite{volckaert2015cloning} and code injection \cite{SalamatJWWF11}. A key design decision of MVE is whether it is achieved in kernel space \cite{osterlund2019kmvx}, in user-space \cite{salamat2009orchestra}, with exploiting hardware features \cite{koning2016secure}, or even through code polymorphism \cite{10.1145/3281662}. Finally, one can neatly exploit the limit case of executing only two variants \cite{maurer2012tachyon,Kim2015}. Notably,  Davi \etal proposed Isomeron \cite{davi2015isomeron}, an approach  for execution-path randomization. Isomeron simultaneously loads the original program and a variant. While the program is running, Isomeron continuously flips a coin to decide which copy of the program should be executed next at the level of function calls. With this strategy, a potential attacker cannot predict whether the original or the variant of a program will execute.
    Researching on MVE in a distributed setting like the Edge \citationneeded has been less researched. Voulimeneas \etal proposed a multivariant execution system by parallelizing the execution of the variants in different machines \cite{voulimeneas2021dmvx} for the sake of efficiency. 
    
    %
\end{usage}


%\begin{usage}{Moving Target Defense(MTD):}
%    \label{usage:mtd}
%    \normalfont
\autoref{usage:randomization} and \autoref{usage:mve} can be categorized as Moving Target Defense strategies. Moving Target Defense for software was first proposed as a collection of techniques that aim to improve the security of a system by constantly moving its vulnerable components \cite{MTDNationalCyberLaep, okhravi2013survey}. Usually, MTD techniques revolve around changing system inputs and configurations to reduce attack surfaces. 
This increases uncertainty for attackers and makes their attacks more difficult. Ultimately, potential attackers cannot hit what they cannot see. 
MTD can be implemented in different ways, including via dynamic runtime platforms \cite{10.1145/3318216.3363338}. 
In the case of \autoref{usage:n-version}, this usage lacks of the time dimension, \ie program variants are not changed from time to time. 

%\end{usage}




\subsection*{Statement of Novelty}


We contribute to Software Diversification for \wasm using Artificial Diversification, for N-Version, Randomization and Multivariant Execution usages (\autoref{usage:n-version}, \autoref{usage:randomization}, \autoref{usage:mve}) for the sake of reliability and security ( \autoref{goal:reliability} and \autoref{goal:security}). 
The primary motivation for our contributions is that we see in \wasm a monoculture problem. If one environment is vulnerable, all the others are vulnerable in the same manner as the same \wasm binary is replicated. 
Besides, the \wasm environment lacks natural diversity \cite{natural_diversity}. Compared to the work of Harrand \etal \citationneeded, in WebAssembly, one could not use preexisting and different program versions to provide diversification. The current limitations on security and the lack of preexisting diversity motivate our work on software diversification as one preemptive mitigation among a wide range of security countermeasures. 


In \autoref{table:sota:comparison} we listed related work on Software Diversification that support our work. The table is composed by the authors and the reference to their work, the source of diversification (natural or artificial), followed by one column for each motivation, strategy and usage (\autoref{goal:reliability},  \autoref{goal:security},  \autoref{strategy:S1},  \autoref{strategy:S2},  \autoref{strategy:S3},  \autoref{strategy:S4},  \autoref{strategy:S5},  \autoref{strategy:S6},  \autoref{strategy:S7},  \autoref{strategy:S8},  \autoref{strategy:S9}, \autoref{strategy:S10}, \autoref{usage:n-version}, \autoref{usage:randomization} and \autoref{usage:mve}). Each cell in the table contains a checkmark if the strategy, the motivation or the usage of the work match the previously mentioned classifications. The rows are sorted by the year of the work in ascending order. The last two rows locate our contributions. 

{
    \renewcommand{\arraystretch}{1.6}
    \newcolumntype{M}{>{\begin{varwidth}{3cm}}l<{\end{varwidth}}} %M is for Maximal column
    \begin{table}[h]
        %\small
        %\begin{sidewaystable}
        \centering
    %\setlength\minrowclearance{1.0pt}
    \resizebox{\textwidth}{!}{

        %\begin{adjustbox}{angle=90}
                \begin{tabular}[t]{ l ||ll|ll|llllll|l|lll||lll||}
Authors & \textbf{ND} & \textbf{AD} & \textbf{\autoref{goal:reliability}} & \textbf{\autoref{goal:security}} & \textbf{\autoref{strategy:S1}} & \textbf{\autoref{strategy:S2}} & \textbf{\autoref{strategy:S3}} & \textbf{\autoref{strategy:S4}} & \textbf{\autoref{strategy:S5}} & \textbf{\autoref{strategy:S6}} & \textbf{\autoref{strategy:S7}} & \textbf{\autoref{strategy:S8}} & \textbf{\autoref{strategy:S9}} & \textbf{\autoref{strategy:S10}} & \textbf{\autoref{usage:n-version}} & \textbf{\autoref{usage:randomization}} & \textbf{\autoref{usage:mve}} \\
\hline
Chew and Song \cite{Chew02mitigatingbuffer} & &\checkmark & & & & & & &\checkmark & & & & & & & & \\
Kc \etal  \cite{Kc03} & &\checkmark & &\checkmark & & & & & & & &\checkmark &\checkmark & & & & \\
Barrantes \etal  \cite{barrantes2003randomized} & &\checkmark & &\checkmark & & & & & & & &\checkmark & & & & & \\
Bhatkar \etal \cite{bhatkar03} & &\checkmark & &\checkmark & &\checkmark &\checkmark & & &\checkmark &\checkmark & & & & & & \\
Bhatkar \etal \cite{bhatkar2005efficient} & &\checkmark & &\checkmark & &\checkmark &\checkmark & & &\checkmark &\checkmark & & & & & & \\
Cox \etal \cite{cox06} &\checkmark & & &\checkmark & & & & & & & & & &\checkmark & & & \\
Bruschi \etal \cite{bruschi2007diversified} & &\checkmark & &\checkmark & & & & & & &\checkmark & & &\checkmark & & & \\
Salamat \etal \cite{salamat2007stopping} & &\checkmark & &\checkmark & & & & &\checkmark & & & & &\checkmark & & & \\
Jacob \etal \cite{jacob2008superdiversifier} & &\checkmark & & &\checkmark & &\checkmark & & &\checkmark & & & & &\checkmark & & \\
Salamat \etal \cite{salamat2009orchestra} & & & & & & & & & & & & & & & & & \\
Amarilli \etal  \cite{amarilli2011can} & & & &\checkmark &\checkmark & & & & & & & &\checkmark & & & & \\
Jackson  \cite{jackson} & &\checkmark & & & & & & & &\checkmark & & & & & & & \\
Maurer \etal \cite{maurer2012tachyon} & & & & & & & & & & & & & & & &\checkmark & \\
Homescu \etal \cite{homescu2013profile} & &\checkmark & & & & & & & &\checkmark & & & & & & & \\
Coppens \etal  \cite{coppens2013feedback} &\checkmark & & & & & & & & & & & & & & & & \\
Agosta \etal  \cite{agosta2015meet} &\checkmark & & &\checkmark & & & & & & & & & & & & & \\
Crane \etal  \cite{crane2015thwarting} & &\checkmark & &\checkmark & &\checkmark & & &\checkmark &\checkmark & & & &\checkmark & & & \\
Davi \etal \cite{davi2015isomeron} & &\checkmark & &\checkmark & & & & & & & & & &\checkmark & &\checkmark & \\
Courouss{\'e} \etal  \cite{courousse2016runtime} & & & &\checkmark & &\checkmark & & & &\checkmark & & &\checkmark &\checkmark & & & \\
Koning \etal \cite{koning2016secure} & & & & & & & & & & & & & & & &\checkmark & \\
Sengupta \etal \cite{10.5555/3091125.3091155} &\checkmark & & &\checkmark & & & & & & & & & &\checkmark & & & \\
Lu \etal \cite{lu2018stopping} & &\checkmark & &\checkmark & & & & & & &\checkmark & & &\checkmark & & &\checkmark \\
Belleville \etal \cite{10.1145/3281662} & & & & & & & & & & & & & & & &\checkmark & \\
Aga \etal \cite{aga2019smokestack} & &\checkmark & &\checkmark & & & & & & &\checkmark & & & & & & \\
Roi \etal \cite{10.1145/3318216.3363338} &\checkmark & & &\checkmark & & & & & & & & & &\checkmark & & & \\
{\"O}sterlund \etal \cite{osterlund2019kmvx} & & & & & & & & & & & & & & & & & \\
Xu \etal \cite{xu2020merr} & &\checkmark & &\checkmark & & & & & & &\checkmark & & & & & & \\
Lee \etal \cite{lee2021savior} & &\checkmark & &\checkmark & & & & & & &\checkmark & & & & & & \\
Voulimeneas \etal \cite{voulimeneas2021dmvx} & &\checkmark & & & & & & & & & & & &\checkmark & & &\checkmark \\

\end{tabular}
        %\end{adjustbox}
    }

        {
            \scriptsize
            $^\dagger$ Notice that LLVM only supports \wasm backend from version 8.0.0
        }

        \caption{The table is composed by the authors and the reference to their work, followed by one column for each strategy and usage ( \autoref{strategy:S1},  \autoref{strategy:S2},  \autoref{strategy:S3},  \autoref{strategy:S4},  \autoref{strategy:S5}, \autoref{usage:randomization} and \autoref{usage:mve}). The last column of the table summarize their technical contribution. Each cell in the table contains a checkmark if the strategy or the usage of the work match the previously mentioned classifications. The rows are sorted by the year of the work in ascending order. The last two rows locate our contributions.}
        \label{table:sota:comparison}
\end{table}
        %\end{sidewaystable}
}

As the table illustrates, we push state of the art in Software Diversification. Our first contribution, CROW \cite{CROW} generates multiple program variants for \wasm using the LLVM pipeline.
It contributes to state of the art in artificially creating diversity for \wasm. While the number of related work for software diversity is large, only one approach has been applied to the context of \wasm. To the best of our knowledge, the closest diversification work on the browsers involving \wasm is the work of Romano \etal \cite{wobfuscator}. They proposed to de-inline JavaScript (\autoref{strategy:S5}) subexpressions and replace them with function calls to \wasm counterparts. The presence of two different engines, one for JS and another for \wasm in the majority of the browser vendors, motivated their work. They empirically demonstrated that malware classifiers could be evaded with this diversification technique. On the other hand, \wasm is a novel technology, and the adoption of defenses for it is still under development \cite{Narayan2021Swivel, johnson2021}.

CROW, extrapolates the idea of superdiversification \cite{jacob2008superdiversifier} for \wasm. CROW works directly with LLVM IR, enabling it to generalize to more languages and CPU architectures, something not possible with the x86-specific approach of previous works.
CROW focuses on the static diversification of software. However, because of the specificities of code execution in the browser, this is not far from being a randomization approach. For example, since \wasm is served at each page refreshment, every time a user asks for a \wasm binary, she can be served a different variant provided by CROW.
It also can be used in fuzzing campaigns \citationneeded to provide reliability. The diversification created by CROW can unleash hidden behaviors in compilers and interpreters. Thanks to CROW, a bug was discovered in the Lucet compiler \footnote{\url{https://www.fastly.com/blog/defense-in-depth-stopping-a-wasm-compiler-bug-before-it-became-a-problem}} 


Moreover, with MEWE \cite{MEWE}, we contribute to the field of Software Diversification at two stages. First, we automatically generate variants of a given program with CROW \cite{CROW}. Second, we randomly select which variant is executed at runtime, creating a multivariant execution scheme that randomizes the observable behaviors at each run of the program. We use the natural redundancy of Edge-Cloud computing architectures to deploy an internet-based MVE.



\begin{comment}

\subsubsection{Automatically}
    Compilers, source -> bin transformation
        set mutators, nop, instruction reordering

    nop
    
    

    compiler flags



    source -> source 

    bin -> bin 
        ASLR, 
\subsubsection{Designed diversity}
\subsubsection{Use of natural diversity}

\todo{ Mendez 2013, Leveson 1986, Collberg 2012 }
\todo{ Koopman and DeVale 1999 }

\subsection{How to use software diversification?}

\subsubsection{1 machine, n-version}



\subsubsection{1 variant per machine}
\todo{randomization}



\todo{Check this, it is the same as the survey paper}



\todo{Add wasmfuscator}




\subsubsection{MTD, Okhravi}





Using this technique, the authors illustrated that some CVE related to specific database engines could be avoided.
s

\subsubsection{multivariant, }





% Statement of novelty





\end{comment}

\subsection*{Conclusions}
In this chapter we presented the background on the \wasm language including our motivation on its security issues and related work enforcing its security.
Software Diversification has been widely researched, not being the case for \wasm. In this chapter, we aim to settle down the foundation to study automatic diversification for \wasm. We stated our contributions to the field of artificial diversity by extending current state of the art. 
Our contributions are obtained by following the methodology described in \autoref{chapter:method}. 
